\lstdefinestyle{myGrammarStyle}{
    basicstyle=\scriptsize\ttfamily, % Reduce font size
    commentstyle=\color{green},
    keywordstyle=\color{blue},
    stringstyle=\color{orange},
    numbers=left, % Line numbers on left
    numberstyle=\tiny\color{gray}, % Line numbers styling
    breaklines=true, % Wrap long lines
    frame=single, % Frame around the code
    framesep=3pt, % Adjust frame separation
    xleftmargin=5pt, % Adjust left margin
    xrightmargin=5pt, % Adjust right margin
    backgroundcolor=\color{yellow!0}, % Background color
    tabsize=2, % Tab size
    captionpos=b, % Caption position bottom
    aboveskip=5pt, % Reduce space above the listing
    belowskip=5pt, % Reduce space below the listing
    linewidth=0.9\linewidth, % Set custom line width to less than text width
    escapeinside={(*@}{@*)}, % for escaping to LaTeX
    % morekeywords={Output, JSON}, % Add specific words to be highlighted
    % keywordstyle=\color{red}\bfseries, % Style for additional keywords
}

\begin{lstlisting}[style=myGrammarStyle, caption= Example Python prompt from the HumanEval dataset ~\cite{athiwaratkun2023multilingual}]
def has_close_elements(numbers: List[float], threshold: float) -> bool:
        """ Check if in given list of numbers, are any two numbers closer to each other than
        given threshold.
        >>> has_close_elements([1.0, 2.0, 3.0], 0.5)
        False
        >>> has_close_elements([1.0, 2.8, 3.0, 4.0, 5.0, 2.0], 0.3)
        True
        """

\end{lstlisting}
\label{prompt:python_prompt}