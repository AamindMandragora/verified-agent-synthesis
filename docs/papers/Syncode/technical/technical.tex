This section describes our main technical contributions and the \Tool{} algorithm. 
% First, we formally define the syntactical decoding problem (Section~\ref{sec:syndecode}). 
% Next, we describe our parsing algorithm (Section~\ref{sec:incparse}). 
% Then we explain our CFG-based masking technique (Section~\ref{sec:mask}) and prove the soundness and completeness of our algorithm (Section~\ref{sec:prop}).
% We finally summarize the \Tool{} implementation (Section~\ref{sec:implementation}) and analyze its time complexity (Section~\ref{sec:time}).  

\subsection{Syntactical Decoding Problem}
\label{sec:syndecode} 

Given a language with grammar $G$, let $\lang(G) \subseteq \alphabets^*$ denote the set of all syntactically valid outputs according to the grammar $G$. 
For a grammar $G$, $\lang_p(G)$ represents the set of all syntactically valid partial outputs. 
If a string $w_1$ belongs to $\lang_p(G)$, then there exists another string $w_2$ such that appending $w_2$ to $w_1$ results in a string that is in the language defined by $G$. Formally,

\begin{definition}[Partial Outputs]
\label{def:gramdecode}
For grammar $G$, $\lang_p(G) \subseteq \alphabets^*$ denotes all syntactically valid partial outputs. Formally, if $w_1 \in \lang_p(G)$ then $\exists w_2 \in \alphabets^*$ such that $w_1.w_2 \in \lang(G)$
\end{definition}

\noindent For a grammar $G$ and a partial output $\partialcode$ belonging to the set of prefix strings $\lang_p(G)$, the syntactical decoding problem aims to determine the set $\vocab_k$ of valid tokens from a finite vocabulary $\vocab$ such that appending any token $t \in \vocab_k$ to $\partialcode$ maintains its syntactic validity according to the grammar $G$. 

\begin{definition}[Syntactical Decoding]
\label{def:gramdecode}
For grammar $G$, given partial output $\partialcode \in \lang_p(G)$ and finite token vocabulary $\vocab \subset \alphabets^*$, the syntactical decoding problem is to compute the set $\vocab_k \subseteq \vocab$ such that for any $t \in \vocab_k, \partialcode.t \in \lang_p(G)$    
\end{definition}

% Synchromesh~\cite{poesia2022synchromesh} solves this problem by iterating over the $\vocab$ and for each $t \in \vocab$ it parses $\partialcode.t$ to check if it is in $\lang_p(G)$. 
% Typically $|V|$ is large ($>30,000$) and thus Synchromesh does a preorder traversal over a trie built on $\vocab$ and optimizes this step.
% Despite this optimization, Synchromesh still needs to evaluate a considerable number of candidate tokens, introducing a large overhead.
% As described earlier, prior works~\cite{poesia2022synchromesh, llamacpp} solve this problem by iterating over the $\vocab$ and for each $t \in \vocab$ it parses $\partialcode.t$ to check if it is in $\lang_p(G)$

% \Tool{} solves this problem through the creation of a novel structure which we call \emph{DFA mask store} offline (Definition~\ref{def:lookup}). 
% For a given grammar $G$ and vocabulary $\vocab$, this mask store is constructed once and can be leveraged across all generations. 
% It efficiently stores masks over the vocabulary.
% \Tool{} breaks down the syntactical decoding problem into two distinct steps.
\mbox{We next present \Tool{}'s key aspects to solve this problem:}
\begin{itemize}[leftmargin=*]
    \item In the initial step, it parses $\partialcode$ and computes the unparsed remainder $r \in \alphabets^*$ along with the acceptable terminal sequences $\accepts$ (Section~\ref{sec:parse}).
    \item In the second step, \Tool{} utilizes $r$, $\accepts$, and the precomputed mask store. This phase involves traversing the DFA and performing a few lookups within the DFA mask store to obtain the set of syntactically valid tokens $t$ capable of extending $\partialcode$ (Section~\ref{sec:mask}). 
    \item Consequently, \Tool{} efficiently computes the set of syntactically valid tokens. We show the soundness and completeness of our approach in Section~\ref{sec:prop}.
    \item We further discuss the theoretical complexity of \Tool{} in Section~\ref{sec:time} and the \Tool{} framework in Section~\ref{sec:framework}.
\end{itemize}
 

\subsection{Parsing Partial Output}
\label{sec:parse}
In this section, we describe the remainder $r$ and accept sequences $\accepts$ returned by the parsing step.

% remainder
\noindent{\bf Remainder.}
\Tool{} uses a lexer to convert $\partialcode$ to sequence of lexical tokens $l_1, l_2 \dots l_f \in \alphabets^*$.  
Each lexical token $l_i$ is associated with a terminal type $\terminal_i$, where $l_i \in \lang(\regex_{\terminal_i})$ ($\regex_{\terminal_i}$ is the regular expression for terminal $\terminal_i$).
\add{
We assume our lexer uses a 1-character lookahead without backtracking. 
This ensures that the lexical types of previous tokens in $\partialcode$ remain unchanged, except for the final token. 
The remainder $r$ represents the suffix of $\partialcode$ that could potentially change its lexical type in future iterations.
Thus the remainder $r$ is assigned such that it is either unlexed because it does not match any terminal, or has been lexed but might undergo a different lexing in subsequent iterations when $\partialcode$ is extended by the LLM by appending tokens.
This assumption is crucial for enabling incremental parsing and ensures that the remainder $r$ remains small, which contributes to reducing overall time complexity.
}
\Tool{} assigns the remainder according to the following two cases:

\begin{description}
    \itemsep0em
    \item \textbf{Case 1: $\partialcode = l_1.l_2 \dots l_f$} 
    Assuming a standard lexer with 1-character lookahead and no backtracking, all lexical tokens $l_1, l_2, \dots, l_{f-1}$ remain unchanged upon extending $\partialcode$. However, the final lexical token $l_f$ may change. 
    For example, in Python partial output in the $k$-th LLM iteration, if the final lexical token is $l_f=$\str{ret} and the language model generates the token \str{urn} in the next iteration, the updated code results in the final lexical token becoming $l_f=$\str{return}. 
    This transition reflects a transformation from an identifier name to a Python keyword in the subsequent iterations. 
    Thus, $r$ is assigned the value $l_f$, i.e., $r=$\str{ret} for k-th iteration in our example.
    
    \item \textbf{Case 2: $\partialcode = l_1.l_2 \dots l_f.u$:} 
    Here, $u \in \alphabets^*$ is the unlexed remainder of $\partialcode$. 
    In this case, considering the 1-character lookahead of the lexer, the types of $l_1, l_2, \dots, l_{f}$ do not change upon extending $\partialcode$. 
    Consequently, $r$ is assigned value $u$ of the suffix that remains unlexed.
\end{description}

\noindent 
\add{
\Tool{} parsing step partitions partial output $\partialcode$ into lexically fixed part $\fixpartialcode$ and remainder $r$.}
Given a sequence $\sequence = \terminal_0, \terminal_1, \dots, \terminal_f$, we simplify notation by using $\lang(\sequence) = \lang(\regex_{\terminal_0} \cdot \regex_{\terminal_1} \dots \regex_{\terminal_f})$ throughout the rest of the paper. 

% Partial parse
\begin{definition}[Partial Parse]
\label{def:pparse}
Given the partial output $\partialcode \in \alphabets^*$, the partial parse function $\partialparse: \alphabets^* \to \allterminals^* \times \alphabets^*$ returns a terminal sequence $\sequence^{\square}$ and remainder $r$ such that $\partialcode = \fixpartialcode.r$ and $\fixpartialcode$ is parsed as $\sequence^{\square}$. i.e. $\fixpartialcode \in \lang(\sequence^{\square})$. 
\end{definition}


% accept sequences
\noindent{\bf Accept Sequences.}
A sentence is a sequence of terminals. 
A grammar $G$ describes a (possibly infinite) set of sentences, that can be derived by using the production rules of the grammar. 
We use $\lang^\allterminals(G) \subseteq \allterminals^*$ to denote the valid sequences of terminals that can be derived from the rules of $G$.
Further, $\lang^\allterminals_p(G)$ denotes all syntactically valid partial sentences of terminals.
Formally, 

\begin{definition}[Partial Sentences]
\label{def:psentence}
We define a set of all syntactically valid partial sentences 
 $\lang^\allterminals_p(G) \subseteq \allterminals^*$ such that $\sequence \in \lang^\allterminals_p(G)$ if and only if $\exists \sequence_1 \in \allterminals^*$ such that $\sequence.\sequence_1 \in \lang^\allterminals(G)$.
\end{definition}
Note that $\lang(G)$ and $\lang_p(G)$ are defined over alphabet $\alphabets$, whereas $\lang^\allterminals(G)$ and $\lang^\allterminals_p(G)$ over terminals $\allterminals$.
Nevertheless, if a program $C$ is parsed to obtain terminal sequence $\sequence$, then $C \in \lang(G)$ is equivalent to $\sequence \in \lang^\allterminals(G)$.
The \Tool{} parsing algorithm obtains $\sequence^{\square} = \terminal_1, \terminal_2 \dots \terminal_f$ by parsing $\partialcode$ \add{corresponding to the parserd part of partial output $\fixpartialcode$.} 
Given a partial sentence $\sequence_{\square}$, an accept sequence is a sequence over $\allterminals$ such that when appended to $\sequence^{\square}$ the result is still a partial sentence.

\begin{definition} [Accept Sequence]
\label{def:acc}
Given partial output $\partialcode \in \lang_p(G)$, and $\sequence^{\square}, r = \partialparse(\partialcode)$,  $\sequence_1 \in \allterminals^*$ is an accept sequence if $\sequence^{\square}.\sequence_1 \in \lang_p^\allterminals(G)$.
\end{definition}
%
Consider a Python partial program $\partialcode =$ \str{def is} and let $\textit{def}, \textit{name}, \textit{lpar}$ and $ \textit{rpar}$ be the terminals in Python grammar. 
we get $\{\textit{def}\},$\str{is} $=\partialparse($\str{def is}$)$, where $\sequence^{\square}=\{\textit{def}\}$ and $r=$\str{is}.
$\sequence_1 = \{\textit{name}, \textit{lpar}, \textit{rpar}\}$ is an accept sequence in this case as the sequence of terminals $\sequence^{\square}.\sequence_1 = \{\textit{def}, \textit{name}, \textit{lpar}, \textit{rpar}\}$ is a valid partial sentence.
The parser state on parsing the partial output $\partialcode$ can be utilized to compute a set of accept sequences denoted as $\accepts$.
The soundness and completeness of the \Tool{} algorithm depend on the length of these accept sequences in $\accepts$.
In theory, using longer accept sequences enhances the precision of the \Tool{} algorithm at the cost of increased computational complexity. 
In Section~\ref{sec:implementation}, we show our method for obtaining 1 and 2-length accept sequences that are efficient and precise in practice.

\subsection{Grammar Mask}
\label{sec:mask}
This section outlines the utilization of the set of acceptable terminal sequences $\accepts$ and the remainder $r$ in the creation of a boolean mask using the DFA mask store which is subsequently used for constraining the LLM output. 
The DFA mask store is constructed offline and makes \Tool{} efficient during the LLM generation. 
%
Given partial output $\partialcode$, our objective is to identify tokens $t \in \vocab$ such that appending them to $\partialcode$ leads to syntactical completion. 
% We approach this problem by utilizing the remainder $r$ and sequences $\accepts$. 
Given remainder $r$ and set of sequences $\accepts$, the goal is to determine whether $r.t$ partially matches the regular expression derived from any of the sequences in $\accepts$.
To characterize the notion of strings partially matching a regular expression, we next introduce the function $\pmatch$.

\begin{definition}[\pmatch]
\label{def:pmatch}
The function $\pmatch$ takes a word $w \in \alphabets^*$, a regular expression $\regex$ and returns a boolean. $\pmatch(w, \regex) = \true$ if either of the following conditions holds:
\begin{enumerate}
    \item $\exists w_1 \in \alphabets^*, w_2 \in \alphabets^+$ such that $w = w_1.w_2 $ and $w_1 \in \lang(\regex)$ or
    \item $\exists w_1 \in \alphabets^*$ such that $w.w_1 \in \lang(\regex)$
\end{enumerate}
\end{definition}

\noindent Thus $\pmatch(w, \regex)$ is true when either a prefix of $w$ matches $\regex$ or $w$ can be extended to match $\regex$.
The consequence of allowing $\pmatch$ to be defined such that it is true even when prefix matches, is that \Tool{} will conservatively accept all tokens for which the prefix matches the accept sequence.
Hence, we overapproximate the precise set of syntactically valid tokens.
We make this choice to ensure that \Tool{} is sound for any length of accept sequences.
Next, we give definitions related to DFAs. 
These definitions are useful for describing the construction of the DFA mask store and proving properties related to its correctness in the \Tool{} algorithm. 
In particular, we first define the live states of DFA. 
We say state $q$ is live if there is a path from $q$ to any final states in $\dfafinal$. Formally,

\begin{definition} [DFA $\live$ states]
\label{def:live}
% Define live states
Given a DFA $\dfa(\dfastates, \alphabets, \transitions, \dfastart, \dfafinal)$, let \mbox{$\live(\dfastates) \subseteq \dfastates$} denote the set of live states such that  
\[
    q \in \live(\dfastates) \text{ iff } \exists w \in \alphabets^* \text{ s.t. } \compute(w, q) \in \dfafinal
\]
\end{definition}

% Define and describe the mask store
\noindent We use $\dfa_\terminal(\dfastates_\terminal, \alphabets_\terminal, \transitions_\terminal, q_0^{\terminal}, \dfafinal_\terminal)$ to denote a DFA corresponding to a terminal $\terminal \in \allterminals$. 
%
Next, we establish the definition of $\dmatch$ for DFA, which is an equivalent concept to $\pmatch$ with regular expressions.
$\dmatch$ is recursively defined such that its computation can be performed by walking over the DFAs of a sequence of terminals.

% Define valid token at a state
\begin{definition} [\dmatch]
\label{def:dmatch}
Given a DFA $\dfa(\dfastates, \alphabets, \transitions, \dfastart, \dfafinal)$, a string $w \in \alphabets^*$, a DFA state $q \in Q$ and any sequence of terminals $\sequence= \{\terminal_{f+1}, \terminal_{f+2} \dots \terminal_{f+d}\}$, $\dmatch(w, q, \sequence) = \true$, if either of the following conditions hold:
\begin{enumerate}
\item $\compute(w, q) \in \live(Q)$ or
\item $\exists w_1 \in \alphabets^*, w_2 \in \alphabets^+$ such that $w_1.w_2 = w$, $\compute(w_1, q) \in F \text{ and } \sequence= \{\} $ or
\item $\exists w_1 \in \alphabets^*, w_2 \in \alphabets^*$ such that $w_1.w_2 = w$, $\compute(w_1, q) \in F$, \\ 
and $\text{\dmatch}(w_2, q_{0}^{\terminal_{f+1}}, \{\terminal_{f+2} \dots \terminal_{f+d}\}) = \true$ where $q_{0}^{\terminal_{f+1}}$ is the start state corresponding to the DFA for $\terminal_{f+1}$
% $w_2$ is valid with respect to $q_{0, \terminal_1}$ and the follow sequence $\{\terminal_2 \dots \terminal_\alpha\}$
\end{enumerate}
\end{definition}

\sloppypar
\noindent Given an accept sequence $\sequence = \{\terminal_{f+1}, \terminal_{f+2} \dots \terminal_{f+d}\} \in \accepts$, our objective is to compute the set of tokens $t \in \vocab$ such that $\pmatch(r.t, \regex_\sequence)$ holds, where $\regex_\sequence = (\regex_{f+1}. \regex_{f+2}. \ldots.\regex_{f+d})$ is the regular expression obtained by concatenating regular expressions for terminals. 
If $\sequence^p$ denotes the sequence $\{\terminal_{f+2}, \dots \terminal_{f+d}\}$, Lemma~\ref{lemma:eq} simplifies this problem to finding $\dmatch(r.t, \dfastart^{\terminal_1}, \sequence^p)$. 
Furthermore, utilizing Lemma~\ref{lemma:dmatch}, this can be further reduced to computing $q = \compute_{\terminal_1}(r, \dfastart^{\terminal_1})$ and $\dmatch(t, q, \sequence^p)$. 
It's important to note that $\dmatch(t, q, \sequence^p)$ does not depend on $\partialcode$ and can be computed offline. 
While the computation of $q$ for $\dmatch(t, q, \sequence^p)$ is relatively inexpensive, evaluating $\dmatch(t, q, \sequence^p)$ can be computationally expensive both offline and online, as it requires considering numerous potential accept sequences offline, and where it needs to iterate over all tokens in $\vocab$ online.
We observe that if we consider sequences of smaller lengths, we can efficiently precompute the set of tokens satisfying $\dmatch(t, q, \sequence^p)$ for all $q, t$ and $\sequence^p$ offline.
We later establish the soundness of \Tool{} when using accept sequences of length at least $1$ (Theorem~\ref{thm:sound}) and completeness for accept sequences of the length greater than maximum length of tokens in the vocabulary (Theorem~\ref{thm:complete}). 
Typically, LLM tokens are small in size, allowing us to obtain these guarantees.
% Building upon these definitions, we establish a crucial lemma that draws a connection between $\pmatch$ and $\dmatch$.

\begin{restatable}{lemma}{eq}
\label{lemma:eq}
Given $\sequence = \{\terminal_{f+1}, \terminal_{f+2} \dots \terminal_{f+d}\}$,  $\sequence^p = \{\terminal_{f+2} \dots \terminal_{f+d}\}$ and $\regex_\sequence = (\regex_{f+1}, \regex_{f+2}, \ldots, \regex_{f+d})$, $\dmatch(w, \dfastart^{\terminal_1}, \sequence^p) \iff \pmatch(w, \regex_\sequence)$.
\end{restatable}
\begin{restatable}{lemma}{dm}
\label{lemma:dmatch}
If $q = \compute_{\terminal}(r, \dfastart^{\terminal})$ and no prefix of $r$ is in $\lang(\terminal)$ i.e. $\nexists w_1 \in \alphabets^*, w_2 \in \alphabets^* \text{ such that } w_1.w_2 = r \text{ and } \compute_\terminal(w_1, \dfastart^{\terminal}) \in F_{\terminal}  $ then $\dmatch(t, q, \sequence) \iff \dmatch(r.t, \dfastart^{\terminal}, \sequence)$.
\end{restatable}
The proofs of both the lemmas are in Appendix~\ref{sec:proofs}.

\begin{figure}[b]
\centering
\includegraphics[width=10cm]{images/dfa_example.png}
\vspace{-.1in}
\caption{DFAs in accept sequence $\sequence = \{\textit{name}, \textit{lpar}, \textit{rpar}\}$ for example. 
The start state, final states, and dead states are in gray, green, and red respectively.
The dashed arrows link the final states of one DFA to the starting state of the next DFA, adhering to condition 3 in Definition~\ref{def:dmatch}. This illustrates the sequential traversal across DFAs during the computation of \dmatch.
} 
\label{fig:dfa}
\vspace{-.1in}
\end{figure}

\noindent \textbf{Illustrative Example:} 
Consider the scenario with $\partialcode =$ \str{def is}, $r=$\str{is}, and an accept sequence $\sequence = \{\textit{name}, \textit{lpar}, \textit{rpar}\}$ in $\accepts$, where $\textit{name}$, $\textit{lpar}$, and $\textit{rpar}$ are terminals in $\allterminals$. 
Our objective is to determine all $t \in \vocab$ such that \str{def is}.t forms a valid partial program. 
This can be achieved by finding tokens $t$ that satisfy $\pmatch(\text{\str{is}}.t, \regex_\sequence)$, where $\regex_\sequence = [a\text{-}z,A\text{-}Z,\_]^*()$.
Let's consider a token $t = \text{\str{\_prime():}}$. We observe that $r.t=$\str{is\_prime():} can be decomposed into \str{is\_prime} ($\textit{name}$), \str{(} ($\textit{lpar}$), \str{)} ($\textit{rpar}$), and \str{:}. 
Consequently, it partially matches $\regex_\sequence$ as defined by $\pmatch$. 
In Figure~\ref{fig:dfa}, we present the DFAs for $\sequence$ used in computing $\dmatch$. 
The reduction $\dmatch(r.t, \dfastart^\textit{name}, {\textit{lpar}, \textit{rpar}}) = \dmatch(\text{\str{is\_prime():}}, \dfastart^\textit{name}, {\textit{lpar}, \textit{rpar}})$ simplifies successively to $\dmatch(\text{\str{():}}, \dfastart^\textit{lpar}, {\textit{rpar}})$, then to $\dmatch(\text{\str{):}}, \dfastart^\textit{rpar}, {})$, and finally to $\dmatch(\text{\str{:}}, q_1^\textit{rpar}, {})$.
As $q_1^\textit{rpar}$ is a final state, according to condition 2 of Definition~\ref{def:dmatch}, $\dmatch(\text{\str{:}}, q_1^\textit{rpar}, {})$ holds true.
% This equivalence arises as we traverse the $\dfa_\textit{name}$ first, consuming \str{is\_prime} and reaching a final state in $F_\textit{name}$. 
Next, we define a mask over vocabulary

\begin{definition}[Vocabulary mask]
\label{def:mask}
Given vocabulary $\vocab \subseteq \alphabets^*$, $m \in \{0, 1\}^{|\vocab|}$ is a mask over the vocabulary. We also use $\set(m) \subseteq \vocab$ to denote the subset represented by $m$.
\end{definition}

\noindent {\bf DFA Mask Store}
% Discuss \alpha
\noindent For an integer $\alpha$, we define a DFA table $\dmap{\alpha}$ as the mask store over the DFA states with $\alpha$ lookahead. 
Given the set of all DFA states $\dfastates_\Omega = \bigcup_{\terminal \in \allterminals} \dfastates_\terminal$, the table stores binary masks of size $|\vocab|$, indicating for token string $t$, for any DFA state $q \in \dfastates_\Omega$ and a sequence of $\alpha$ terminals $\sequence_\alpha$ if $\dmatch(t, q, \sequence_\alpha) = \true$. 
The lookahead parameter $\alpha$ signifies the number of subsequent terminals considered when generating the mask stored in the table.
Choosing a larger value for $\alpha$ enhances the precision of \Tool{} algorithm, but it comes at the cost of computing and storing a larger table. 
We next formally define the DFA mask store,
\begin{definition}[DFA mask store]
\label{def:lookup}
For an integer $\alpha$, the DFA mask store $\dmap{\alpha}$ is a function defined as $\dmap{\alpha}: \dfastates_\Omega \times \allterminals^{\alpha} \to \{0, 1\}^{|\vocab|}$, where $\dfastates_\Omega = \bigcup_{\terminal \in \allterminals} \dfastates_\terminal$ represents the set of all DFA states and $\allterminals^{\alpha}$ is a set of $\alpha$-length terminal sequences. 
% Let $\textit{id}(t)$ denote the enumerated index of a token $t \in \vocab$. 
Then $\dmap{\alpha}(q, \sequence) = m$ is a binary mask such that $t \in \set(m)$ if $\dmatch(t, q, \sequence)$
\end{definition}

% \begin{wrapfigure}{R}{0.5\textwidth}
\begin{minipage}{0.5\textwidth}
% \vspace{-.55in}

\begin{algorithm}[H] 
\small
\caption{\Tool{} Generation}
\label{alg:main}
% %
\textbf{Inputs:} $M$: LLM, $\tokenizer$: tokenizer, $C_0$: input prompt, $n_\textit{max}$: maximum generated tokens, $D$: decoding strategy
\begin{algorithmic}[1]
\Function{MaskedGenerate}{$M$, $\tokenizer$, $C_0$, $n_\textit{max}$, $D$}
\State $\curtokens \gets \text{Tokenize}(\tokenizer, C_0)$
\For{$i \in \{1, \dots n_\textit{max} \}$}
\State $\textit{scores} \gets M(\curtokens)$
\State $C_k \gets \text{decode}(\tokenizer, \curtokens)$
\State $\accepts, r \gets \text{Parse}(C_k)$
\label{line:acc}
\State $m \gets \text{GrammarMask}(\accepts, r)$
\label{line:gm}
\State $\textit{scores} \gets m \odot \textit{scores}$
\State $t_i \gets D(\textit{scores})$
\If{$t_i = EOS$}
\State break
\EndIf
\State $T_\textit{cur} \gets \text{append}(T_\textit{cur}, t_i)$
% \State $\curtokens \gets \text{append}(\curtokens, t_i)$
\EndFor
\State $\text{output} \gets \text{decode}(\tokenizer, \curtokens)$
\State \Return output
\EndFunction
\end{algorithmic}
\end{algorithm}
\vspace{-0.4in}

\end{minipage}
\end{wrapfigure}

For our illustrative example if $m = \dmap{2}(q_1^\textit{name}, \{\textit{lpar}, \textit{rpar}\})$ then $t=$\str{\_prime():} should be contained in $\set(m)$.
The grammar mask for a set of accept sequences $\accepts$ can be computed by combining masks for each $\sequence \in \accepts$. 
% We describe the grammar mask computation algorithm in Appendix~\ref{sec:compmask}.
The DFA mask store $\dmap{0}$ maps each DFA state to all tokens such that they  $\pmatch$ without considering any following accept sequence (0-length sequence). 
In this case, the table maps each state with a single mask denoting the tokens that match the regular expression of the corresponding DFA. 

\noindent {\bf Computing Grammar Mask}
\label{sec:compmask}
\begin{wrapfigure}{R}{0.53\textwidth}
\vspace{-.34in}
\begin{minipage}{0.55\textwidth}

\begin{algorithm}[H]
\caption{Computing Grammar Mask}
\label{alg:grammar}
%
\textbf{Inputs:} $\accepts$: set of accept sequences, $r$: remainder
\begin{algorithmic}[1]
\Function{GrammarMask}{$\accepts, r$}
\State $m \gets \{\}$
\For{$\sequence \in \accepts$}
\State $\terminal_1 \gets \sequence[0]$
\State $q_r \gets \compute(q_0^{\terminal_1}, r)$
% \For{$q \in \dfastates_{\terminal_1}$}
\If{$q_r \in \live(\dfastates_{\terminal_1})$}
\State $\Pi \gets \textit{len}(\sequence)-1$
\State $m \gets m \cup \big( \dmap{\Pi}(q_r, \sequence[1:])\big)$
\EndIf
% \EndFor
\EndFor
\State \Return $m$
\EndFunction
\end{algorithmic}
\end{algorithm}

\end{minipage}
\vspace{-0.2in}
\end{wrapfigure}
% Example \dmap{0}
% This approach is equivalent to the one used by \cite{willard2023efficient} for regular expression guided generation. 
% The current parsers can easily compute acceptable sequences of terminals with a length of 2 from partial output. 
% We note that $\pmatch$ $r.t$ with a 2-length sequence is equivalent to $\dmatch$ with a 1-length sequence, as stated in Lemma~\ref{lemma:eq}. 
% Consequently, in our work, we opt for $\dmap{0}$ and $\dmap{1}$ since we have observed empirically that this combination is sufficient for producing syntactically valid outputs.
%
The mask store is constructed offline by enumerating all DFA states $\dfastates_\Omega $, considering all possible terminals in $\allterminals$, and all tokens in $\vocab$. 
The DFA mask store depends on the set of terminals $\allterminals$ and the model's vocabulary $\vocab$. 
As a result, a unique mask store is created for each grammar and tokenizer combination, and to enhance efficiency, we cache and reuse this table for future inferences. 

% \noindent{\bf Computing Grammar Mask.}
Algorithm~\ref{alg:grammar} presents our approach for computing the grammar mask during LLM generation.
It computes a grammar mask based on the sets of current accept sequences $\accepts$, and the remainder string ($r$). 
It iterates over $\accepts$, considering each sequence $\sequence$.
The algorithm initializes an empty mask $m$. 
It iterates over each acceptable sequence, considering the first terminal $\terminal_1$ in each. 
It computes the resulting state $q_r$ by processing $\terminal_1$ from an initial state $q_0^{\terminal_1}$ and the remainder string $r$. 
If $q_r$ is in a live state, the algorithm updates the grammar mask by unifying the mask cached in $\dmap{\alpha}$. 

\subsection{Soundness and Completeness}
\label{sec:prop}
This section establishes the soundness and completeness of the \Tool{} algorithm.
Algorithm~\ref{alg:main} presents the LLM generation algorithm with \Tool{}. It takes as inputs an LLM represented by $M$, a tokenizer denoted by $\tokenizer$, an input prompt string $C_0$, the maximum number of generated tokens $n_\textit{max}$, and a base decoding strategy $D$. The algorithm begins by tokenizing the input prompt using the tokenizer. 
It then iteratively generates tokens using the LLM, decodes the current token sequence, and performs parsing to obtain acceptable terminal sequences $\accepts$, and a remainder $\remainder$ (line~\ref{line:acc}). 
A grammar mask is applied to the logit scores based on these values (line~\ref{line:gm}). 
The algorithm subsequently selects the next token using the decoding strategy, and if the end-of-sequence token (EOS) is encountered, the process terminates. 
The final decoded output is obtained, incorporating the generated tokens, and is returned as the result of the MaskedGenerate algorithm.

Given partial output $\partialcode \in \lang_p(G)$, \Tool{} generates a corresponding mask $m$. 
If, for a token $t \in \vocab$, the concatenation $\partialcode.t$ results in a syntactically valid partial output, i.e. $\partialcode.t \in \lang_p(G)$, our soundness theorem ensures that $t$ is indeed a member of the set defined by the generated mask $m$. 
The subsequent theorem formally states this soundness property.

\begin{restatable}{theorem}{Sound}
\label{thm:sound}
Let $\partialcode \in \lang_p(G)$ be the partial output and any integer $d \geq 1$, let $\accepts_d \subseteq \allterminals^{d}$ contain all possible accept terminal sequences of length $d$ and $r \in \alphabets^*$ denote the remainder. 
If $m = \text{GrammarMask}(\accepts, r)$ then for any $t \in \vocab$, if $\partialcode . t \in \lang_p(G)$ then $t \in \set(m)$
\end{restatable}
The proof of the theorem is in Appendix~\ref{sec:proofs}.

Next, we give a definition that establishes a partial order on sets of terminal sequences, where given two sets $\accepts_1$ and $\accepts_2$, we say sets $\accepts_1 \greater \accepts_2$ if every sequence in $\accepts_2$ has a prefix in $\accepts_1$.

\begin{definition}[$\greater$]
We define a partial order $\greater$ on set of terminal sequences $\mathcal{P}(\allterminals^*)$ such that $\accepts_1 \greater \accepts_2$ when $\forall \sequence_2 \in \accepts_2 \exists \sequence_1 \in \accepts_1 \exists \sequence_3 \in \allterminals^*  $ s.t. $\sequence_2 = \sequence_1.\sequence_3$
\end{definition}

\begin{wrapfigure}{R}{0.5\textwidth}
\begin{minipage}{0.5\textwidth}
% \vspace{-.55in}

\begin{algorithm}[H] 
\small
\caption{\Tool{} Generation}
\label{alg:main}
% %
\textbf{Inputs:} $M$: LLM, $\tokenizer$: tokenizer, $C_0$: input prompt, $n_\textit{max}$: maximum generated tokens, $D$: decoding strategy
\begin{algorithmic}[1]
\Function{MaskedGenerate}{$M$, $\tokenizer$, $C_0$, $n_\textit{max}$, $D$}
\State $\curtokens \gets \text{Tokenize}(\tokenizer, C_0)$
\For{$i \in \{1, \dots n_\textit{max} \}$}
\State $\textit{scores} \gets M(\curtokens)$
\State $C_k \gets \text{decode}(\tokenizer, \curtokens)$
\State $\accepts, r \gets \text{Parse}(C_k)$
\label{line:acc}
\State $m \gets \text{GrammarMask}(\accepts, r)$
\label{line:gm}
\State $\textit{scores} \gets m \odot \textit{scores}$
\State $t_i \gets D(\textit{scores})$
\If{$t_i = EOS$}
\State break
\EndIf
\State $T_\textit{cur} \gets \text{append}(T_\textit{cur}, t_i)$
% \State $\curtokens \gets \text{append}(\curtokens, t_i)$
\EndFor
\State $\text{output} \gets \text{decode}(\tokenizer, \curtokens)$
\State \Return output
\EndFunction
\end{algorithmic}
\end{algorithm}
\vspace{-0.4in}

\end{minipage}
\end{wrapfigure}
We further state the lemma that shows the relation in the grammar masks generated by two accept sequences satisfying relation $\greater$.

\begin{restatable}{lemma}{Porder}
\label{lemma:porder}
Given $\accepts_1$ and $\accepts_2$ are set of accept sequences such that $\accepts_1 \greater \accepts_2$ and $m_1 = \text{GrammarMask}(\accepts_1, r)$ and $m_2 = \text{GrammarMask}(\accepts_2, r)$ then $\set(m_2) \subseteq \set(m_1)$
\end{restatable}
The proof of the lemma is in Appendix~\ref{sec:proofs}.

Theorem~\ref{thm:sound} proves soundness for accept sequences $\accepts_d$ of length $d$, while Lemma~\ref{lemma:porder} extends this proof to any set of accept sequences $\accepts$ where $\accepts \greater \accepts_d$. 
Our implementation, employing sequences of varying lengths, can be proven sound based on this extension.
% Say motivation and pretext for the completeness

The completeness theorem ensures that, under specified conditions, each token $t \in \set(m)$ guarantees $\partialcode.t$ as a syntactically valid partial output.
An implementation of \Tool{} with a short length of accept sequences although sound, may not guarantee completeness.  
To illustrate, let's take the example where $\sequence = \terminal_{f+1}, \terminal_{f+2} \in \accepts$ with simple singleton regular expressions $\regex_{\terminal_{f+1}} =$ \str{(} and $\regex_{\terminal_{f+2}} =$ \str{(}. 
In this case, our algorithm conservatively treats all tokens $t \in \vocab$ as syntactically valid, whenever \str{((} is a prefix of those tokens (e.g., \str{(((}, \str{(()})) even though some tokens may not meet syntactic validity. 
However, by assuming that the accept sequences are long enough, we can establish the completeness of the approach. 
\begin{restatable}{theorem}{Complete}
\label{thm:complete}
Let $\partialcode \in \lang_p(G)$ be the partial output, let $\accepts_d \subseteq \allterminals^{d}$ contain all possible accept terminal sequences of length $d$ and $r \in \alphabets^*$ denote the remainder. 
Suppose for any $t \in \vocab, d > \textit{len}(t)$ and $m = \text{GrammarMask}(\accepts_d, r)$ such that $t \in \set(m)$ then $\partialcode.t \in \lang_p(G)$
\end{restatable}

The proof of the theorem is in Appendix~\ref{sec:proofs}. 
While our completeness theorem ensures the \Tool{} consistently extends syntactically correct partial outputs, it does not guarantee termination with a correct and complete output. 
The focus of the theorem is on generating syntactically valid partial outputs, and the theorem does not address whether the process converges to a syntactically correct whole output. 
Termination considerations go beyond the completeness theorem's scope.



\subsection{\Tool{} Implementation}
\label{sec:implementation}
\add{
\noindent \textbf{Base LR parser: }
Bottom-up LR parsers, including LR(1) and LALR(1) parsers, process terminals generated from the lexical analysis of the code sequentially and perform shift or reduce operations~\cite{aho86}. 
LR($\kappa$) parsers have the immediate error detection property, ensuring they do not perform shift or reduce operations if the next input $\kappa$ terminals on the input tape is erroneous~\cite{10.1145/356628.356629}.
Consequently, every entry in the parsing table corresponding to $\kappa$ terminals that maps to a shift or reduce operation indicates that the terminal is acceptable. 
This property allows us to use LR(1) parsing tables to efficiently compute accept sequences at any intermediate point, making them preferable for \Tool{} applications. 
Thus, computing acceptable terminals with LR(1) parsers has a complexity of $O(|\allterminals|)$.
Although LALR(1) parsers are more commonly used due to their smaller memory requirements and faster construction, computing acceptable terminals with them requires iterating over all terminals leading to a complexity of $O(T_\parser \cdot |\allterminals|)$ due to the need for multiple reduce operations before confirming the validity of each terminal. 
Furthermore, while for $\kappa > 1$, LR($\kappa$) parsers can compute accept sequences of length $\kappa$ immediately, they incur extremely high memory requirements. 
Additionally, while we can use LL($\kappa$) parsing tables to compute the next $\kappa$ accept terminals, LR($\kappa$) parsers offer a higher degree of parsing power. 
Therefore, we employ LR parsers in \Tool{}.
Our evaluation indicates that LR(1) parsers suffice for eliminating most syntax errors, making them a practical choice for \Tool{}. 
}
We discuss how the implementation of how parsing is performed \emph{incrementally} to obtain the accept sequences and remainder in the Appendix~\ref{sec:incparse}.

\noindent \textbf{Accept Sequences:}
In our implementation, we focus on generating accept sequences of length 1 or 2, as they can be efficiently obtained from LR(1) parser. 
While this approach incurs some loss of precision, it leads to sound but incomplete syntactical decoding. 
Further, our evaluation demonstrates that this strategy is efficient and precise in practical scenarios.
We note that $\pmatch$ $r.t$ with a 2-length sequence is equivalent to $\dmatch$ with a 1-length sequence, as stated in Lemma~\ref{lemma:eq}. 
Consequently, in our work, we precompute mask stores $\dmap{0}$ and $\dmap{1}$.
On parsing the partial output $\partialcode$, the parser state \add{of LR(1) parsers} can be used to directly obtain syntactically acceptable terminals for the current completion ($\curaccepts$) and the next completion ($\nextaccepts$). 
We utilize $\curaccepts$ and $\nextaccepts$ to construct the accept sequences $\accepts$, considering two cases:

% \begin{description}
%     \itemsep0em
%     \item 
    
    \textbf{Case 1: $\partialcode = l_1.l_2 \dots l_f$:} Let $\terminal_f$ represent the type of the final lexical token. 
    In many instances, a token may be extended in the subsequent generation step, such as when an identifier name grows longer or additional words are appended to a comment. 
    In those cases if $\nextaccepts = {\terminal_1^1, \terminal_2^1, \dots, \terminal_n^1}$, we include all 2-length sequences $\{\terminal_f, \terminal_i^1\}$ for each $i$. 
    As previously discussed, the type of the final lexical token may change from $\terminal_f$. 
    Consequently, when $\curaccepts = \{\terminal_1^0, \terminal_2^0, \dots, \terminal_n^0\}$, we add 1-length sequences $\sequence_i$ for each terminal sequence $\{\terminal_i\}$ from $\curaccepts$, excluding $\terminal_f$. 
    This method ensures the generation of sequences accounting for potential extensions of the same token and changes in the type of the final lexical token.

    % \item 
    \textbf{Case 2 $\partialcode = l_1.l_2 \dots l_f.u$:} In this scenario, the current terminal is incomplete, leading to a lack of information about subsequent terminals. 
    Consequently, when $\nextaccepts = \{\terminal_1, \terminal_2, \dots, \terminal_n\}$, we define $\accepts$ as a set of sequences: $\{\sequence_1, \sequence_2, \dots, \sequence_n\}$, where each $\sequence_i$ corresponds to a single terminal sequence $\{\terminal_i\}$ from $\nextaccepts$. 
    Specifically, $\sequence_1 = \{\terminal_1\}$, $\sequence_2 = \{\terminal_2\}$, and so forth. 
    % This approach ensures the generation of accept sequences based on the available information for subsequent terminals when the current one is incomplete.
% \end{description}

\subsection{Time Complexity}
\label{sec:time}
In this section, we analyze the time complexity of the \Tool{} algorithm. 
We focus on the cost of creating the mask at each iteration of the LLM generation loop.
The key computations involved in this process are the parsing carried out by the incremental parser to compute $\accepts$ and the lookup/unification operations performed through the DFA mask store.

The incremental parser parses $O(1)$ new tokens at each iteration and computes $\accepts$. 
Let $T_A$ represent the time taken by the parser to compute the accepted terminals and $T_\parser$ denote the time the parser takes to parse a new token and update the parser state. 
Hence, in each iteration, the parser consumes $O(T_A+T_\parser)$ time to generate $\accepts$.
%
The DFA mask store lookup involves traversing $|\accepts|$ DFA sequences, with the number of steps in this walk bounded by the length of the remainder $r$. 
As $\accepts$ can have a maximum of $|\allterminals|$ sequences, the DFA walk consumes $O(|\allterminals| \cdot \len(r))$ time.
We employ a hashmap to facilitate efficient lookups at each DFA node, ensuring that all lookups take constant time. Consequently, this step takes $O(|\allterminals|)$ time. 
Let $T_\cup$ denote the time taken for computing the union of binary masks. 
With potentially $|\allterminals|$ union operations to be performed, the mask computation takes $O(T_\cup \cdot |\allterminals|)$ time.
Therefore, the overall time complexity at each step during generation is given by $O(T_A + T_\parser + |\allterminals| \cdot \len(r) + T_\cup \cdot |\allterminals|)$. 

For an incremental LR(1) parser, the complexity of our algorithm at each step of LLM token generation is $O(|\allterminals| \cdot \len(r) + T_\cup \cdot |\allterminals|)$. 
\add{With our lexer assumption, we ensure that} the remainder $r$ is small, allowing us to further simplify our complexity analysis to $O(T_\cup \cdot |\allterminals|)$ by treating $\len(r)$ as constant.
% Additionally, all these computations have the potential for parallelization during LLM generation, but this aspect is deferred to future work.

\noindent \textbf{Offline cost: } The cost of computing the mask store $\dmap{\alpha}$ offline involves considering all DFA states $q \in Q_\Omega$, all possible terminal sequences of length $\alpha$, and all tokens $t \in \vocab$. 
Given that we need to traverse the DFA for $\len(t)$ steps for each entry in the store, the time complexity for computing the mask store is $O(\textit{max}_{t \in \vocab}(\len(t)).|Q_\Omega|.|\vocab|.|\allterminals|^\alpha)$. 
Typically, $\len(t)$ is small, allowing us to simplify this to $O(|Q_\Omega|.|\vocab|.|\allterminals|^\alpha)$. 
In our implementation, the use of $\dmap{0}$ and $\dmap{1}$ results in a cost of $O(|Q_\Omega|.|\vocab|.|\allterminals|)$.
The size of $|Q_\Omega|$ depends on the complexity of regular expressions for the terminals, which may vary for each grammar. 
However, as demonstrated in our evaluation section, these mask stores can be computed within 10 minutes for each combination of grammar and LLM. 
This computation is a one-time cost that can be amortized over all generations performed for the given LLM and grammar.

\subsection{SynCode Framework}
\label{sec:framework}

\begin{figure}[tb]
\centering
\includegraphics[width=13cm]{images/example_framework.png}
\vspace{-.1in}
\caption{The upper section displays erroneous output from a standard LLM generation, failing to produce the intended JSON format. The lower segment showcases the fix achieved through the use of the \Tool{} framework.
} 
\label{fig:dfa}
\vspace{-.1in}
\end{figure}

Figure~\ref{fig:dfa} shows how \Tool{} framework can be used in practice by selecting a grammar. 
We next discuss other important features of the framework.

\noindent \textbf{Adding a New Grammar.} 
Our Python-based \Tool{} framework is shipped with several built-in grammars such as JSON, Python, Go, etc. 
A user can apply \Tool{} for arbitrary grammar by providing the grammar rules in EBNF syntax with little effort.
The grammar needs to be unambiguous LALR(1) or LR(1) grammar for using the respective base parsers.
% The power of the LALR(1) parser is sufficient for most mainstream formal languages~\cite{10.5555/42212}.

\noindent \textbf{Ignore Terminals.}
Our EBNF syntax adopted from Lark allows one to provide \emph{ignore terminals} as part of the grammar. 
Lark ignores those terminals while parsing. 
In the case of Python, this includes \emph{comments} and \emph{whitespaces}. 
\Tool{} handles these ignore terminals by adding a trivial 1-length accept sequence for each of these ignore terminals.

\noindent \textbf{Parsers.}
\Tool{} supports both LALR(1) and LR(1) as base parsers. 
We adapt Lark's~\cite{lark} LALR(1) parser generator for \Tool{}. 
Since Lark does not implement the LR(1) parser generator, we implemented the LR(1) parser generator on top of the Lark.
The generation of LR(1) parser which is performed offline may take longer time compared to the LALR(1) parser (e.g., up to 2 mins for our Python grammar), however, it is more efficient at inference time in computing the accept sequences.
Further, since the Lark-generated parser is non-incremental, we build the incremental parser on top of it by caching the parser state as described in Appendix~\ref{sec:incparse}.

\noindent \textbf{Non-CFG Fragments of PLs.}
\Tool{} can handle non-context-free fragments of PLs, such as \emph{indentation} in Python and end-of-scope markers in Go.
To support languages with indentation, such as Python and YAML, \Tool{} has a mechanism that tracks acceptable indentation for the next token, effectively masking tokens that violate indentation constraints at a given point.
This indentation constraint feature can be enabled with any new grammar. 
Similarly, for handling other custom parsing rules beyond CFGs, users can add additional constraints to the generation by overriding specific \Tool{} functions. 
For instance, in Go, semicolons are optional and may be automatically inserted at the end of non-blank lines. 
Implementing such constraints in \Tool{} programmatically requires minimal effort.
However, \Tool{} currently does not support enforcing semantic constraints. (e.g, if a variable in a program is defined before it is used.)

% \noindent \textbf{Opportunistic Mode.}
% \add{
% Additionally, \Tool{} incorporates a form of optimization also present in~\cite{beurerkellner2024guiding, llamacpp}, referred to as opportunistic masking. 
% Instead of initially computing the mask, as outlined in Algorithm~\ref{alg:main}, we first execute the decode step of the LLM and use the parser to examine the token suggested by the model. 
% Only if the token is incorrect\fTBD{how wo we know the overhead is going to be better this way than the other?} do we proceed to compute the remainder of the mask, thus allowing the LLM to direct the decoding process.\fTBD{Where do we use it in our experiments? Do we show this optimization is useful/nencessary? Does it perform better in our sys than in those from the 2 refs?}
% }

% \noindent \textbf{Under-approximation.}
% \add{
% Write here!
% }
