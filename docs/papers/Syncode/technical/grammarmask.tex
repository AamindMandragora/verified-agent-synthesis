\subsection{Grammar Mask}
\label{sec:mask}
This section outlines the utilization of the set of acceptable terminal sequences $\accepts$ and the remainder $r$ in the creation of a boolean mask using the DFA mask store which is subsequently used for constraining the LLM output. 
The DFA mask store is constructed offline and makes \Tool{} efficient during the LLM generation. 
%
Given partial output $\partialcode$, our objective is to identify tokens $t \in \vocab$ such that appending them to $\partialcode$ leads to syntactical completion. 
% We approach this problem by utilizing the remainder $r$ and sequences $\accepts$. 
Given remainder $r$ and set of sequences $\accepts$, the goal is to determine whether $r.t$ partially matches the regular expression derived from any of the sequences in $\accepts$.
To characterize the notion of strings partially matching a regular expression, we next introduce the function $\pmatch$.

\begin{definition}[\pmatch]
\label{def:pmatch}
The function $\pmatch$ takes a word $w \in \alphabets^*$, a regular expression $\regex$ and returns a boolean. $\pmatch(w, \regex) = \true$ if either of the following conditions holds:
\begin{enumerate}
    \item $\exists w_1 \in \alphabets^*, w_2 \in \alphabets^+$ such that $w = w_1.w_2 $ and $w_1 \in \lang(\regex)$ or
    \item $\exists w_1 \in \alphabets^*$ such that $w.w_1 \in \lang(\regex)$
\end{enumerate}
\end{definition}

\noindent Thus $\pmatch(w, \regex)$ is true when either a prefix of $w$ matches $\regex$ or $w$ can be extended to match $\regex$.
The consequence of allowing $\pmatch$ to be defined such that it is true even when prefix matches, is that \Tool{} will conservatively accept all tokens for which the prefix matches the accept sequence.
Hence, we overapproximate the precise set of syntactically valid tokens.
We make this choice to ensure that \Tool{} is sound for any length of accept sequences.
Next, we give definitions related to DFAs. 
These definitions are useful for describing the construction of the DFA mask store and proving properties related to its correctness in the \Tool{} algorithm. 
In particular, we first define the live states of DFA. 
We say state $q$ is live if there is a path from $q$ to any final states in $\dfafinal$. Formally,

\begin{definition} [DFA $\live$ states]
\label{def:live}
% Define live states
Given a DFA $\dfa(\dfastates, \alphabets, \transitions, \dfastart, \dfafinal)$, let \mbox{$\live(\dfastates) \subseteq \dfastates$} denote the set of live states such that  
\[
    q \in \live(\dfastates) \text{ iff } \exists w \in \alphabets^* \text{ s.t. } \compute(w, q) \in \dfafinal
\]
\end{definition}

% Define and describe the mask store
\noindent We use $\dfa_\terminal(\dfastates_\terminal, \alphabets_\terminal, \transitions_\terminal, q_0^{\terminal}, \dfafinal_\terminal)$ to denote a DFA corresponding to a terminal $\terminal \in \allterminals$. 
%
Next, we establish the definition of $\dmatch$ for DFA, which is an equivalent concept to $\pmatch$ with regular expressions.
$\dmatch$ is recursively defined such that its computation can be performed by walking over the DFAs of a sequence of terminals.

% Define valid token at a state
\begin{definition} [\dmatch]
\label{def:dmatch}
Given a DFA $\dfa(\dfastates, \alphabets, \transitions, \dfastart, \dfafinal)$, a string $w \in \alphabets^*$, a DFA state $q \in Q$ and any sequence of terminals $\sequence= \{\terminal_{f+1}, \terminal_{f+2} \dots \terminal_{f+d}\}$, $\dmatch(w, q, \sequence) = \true$, if either of the following conditions hold:
\begin{enumerate}
\item $\compute(w, q) \in \live(Q)$ or
\item $\exists w_1 \in \alphabets^*, w_2 \in \alphabets^+$ such that $w_1.w_2 = w$, $\compute(w_1, q) \in F \text{ and } \sequence= \{\} $ or
\item $\exists w_1 \in \alphabets^*, w_2 \in \alphabets^*$ such that $w_1.w_2 = w$, $\compute(w_1, q) \in F$, \\ 
and $\text{\dmatch}(w_2, q_{0}^{\terminal_{f+1}}, \{\terminal_{f+2} \dots \terminal_{f+d}\}) = \true$ where $q_{0}^{\terminal_{f+1}}$ is the start state corresponding to the DFA for $\terminal_{f+1}$
% $w_2$ is valid with respect to $q_{0, \terminal_1}$ and the follow sequence $\{\terminal_2 \dots \terminal_\alpha\}$
\end{enumerate}
\end{definition}

\sloppypar
\noindent Given an accept sequence $\sequence = \{\terminal_{f+1}, \terminal_{f+2} \dots \terminal_{f+d}\} \in \accepts$, our objective is to compute the set of tokens $t \in \vocab$ such that $\pmatch(r.t, \regex_\sequence)$ holds, where $\regex_\sequence = (\regex_{f+1}. \regex_{f+2}. \ldots.\regex_{f+d})$ is the regular expression obtained by concatenating regular expressions for terminals. 
If $\sequence^p$ denotes the sequence $\{\terminal_{f+2}, \dots \terminal_{f+d}\}$, Lemma~\ref{lemma:eq} simplifies this problem to finding $\dmatch(r.t, \dfastart^{\terminal_1}, \sequence^p)$. 
Furthermore, utilizing Lemma~\ref{lemma:dmatch}, this can be further reduced to computing $q = \compute_{\terminal_1}(r, \dfastart^{\terminal_1})$ and $\dmatch(t, q, \sequence^p)$. 
It's important to note that $\dmatch(t, q, \sequence^p)$ does not depend on $\partialcode$ and can be computed offline. 
While the computation of $q$ for $\dmatch(t, q, \sequence^p)$ is relatively inexpensive, evaluating $\dmatch(t, q, \sequence^p)$ can be computationally expensive both offline and online, as it requires considering numerous potential accept sequences offline, and where it needs to iterate over all tokens in $\vocab$ online.
We observe that if we consider sequences of smaller lengths, we can efficiently precompute the set of tokens satisfying $\dmatch(t, q, \sequence^p)$ for all $q, t$ and $\sequence^p$ offline.
We later establish the soundness of \Tool{} when using accept sequences of length at least $1$ (Theorem~\ref{thm:sound}) and completeness for accept sequences of the length greater than maximum length of tokens in the vocabulary (Theorem~\ref{thm:complete}). 
Typically, LLM tokens are small in size, allowing us to obtain these guarantees.
% Building upon these definitions, we establish a crucial lemma that draws a connection between $\pmatch$ and $\dmatch$.

\begin{restatable}{lemma}{eq}
\label{lemma:eq}
Given $\sequence = \{\terminal_{f+1}, \terminal_{f+2} \dots \terminal_{f+d}\}$,  $\sequence^p = \{\terminal_{f+2} \dots \terminal_{f+d}\}$ and $\regex_\sequence = (\regex_{f+1}, \regex_{f+2}, \ldots, \regex_{f+d})$, $\dmatch(w, \dfastart^{\terminal_1}, \sequence^p) \iff \pmatch(w, \regex_\sequence)$.
\end{restatable}
\begin{restatable}{lemma}{dm}
\label{lemma:dmatch}
If $q = \compute_{\terminal}(r, \dfastart^{\terminal})$ and no prefix of $r$ is in $\lang(\terminal)$ i.e. $\nexists w_1 \in \alphabets^*, w_2 \in \alphabets^* \text{ such that } w_1.w_2 = r \text{ and } \compute_\terminal(w_1, \dfastart^{\terminal}) \in F_{\terminal}  $ then $\dmatch(t, q, \sequence) \iff \dmatch(r.t, \dfastart^{\terminal}, \sequence)$.
\end{restatable}
The proofs of both the lemmas are in Appendix~\ref{sec:proofs}.

\begin{figure}[b]
\centering
\includegraphics[width=10cm]{images/dfa_example.png}
\vspace{-.1in}
\caption{DFAs in accept sequence $\sequence = \{\textit{name}, \textit{lpar}, \textit{rpar}\}$ for example. 
The start state, final states, and dead states are in gray, green, and red respectively.
The dashed arrows link the final states of one DFA to the starting state of the next DFA, adhering to condition 3 in Definition~\ref{def:dmatch}. This illustrates the sequential traversal across DFAs during the computation of \dmatch.
} 
\label{fig:dfa}
\vspace{-.1in}
\end{figure}

\noindent \textbf{Illustrative Example:} 
Consider the scenario with $\partialcode =$ \str{def is}, $r=$\str{is}, and an accept sequence $\sequence = \{\textit{name}, \textit{lpar}, \textit{rpar}\}$ in $\accepts$, where $\textit{name}$, $\textit{lpar}$, and $\textit{rpar}$ are terminals in $\allterminals$. 
Our objective is to determine all $t \in \vocab$ such that \str{def is}.t forms a valid partial program. 
This can be achieved by finding tokens $t$ that satisfy $\pmatch(\text{\str{is}}.t, \regex_\sequence)$, where $\regex_\sequence = [a\text{-}z,A\text{-}Z,\_]^*()$.
Let's consider a token $t = \text{\str{\_prime():}}$. We observe that $r.t=$\str{is\_prime():} can be decomposed into \str{is\_prime} ($\textit{name}$), \str{(} ($\textit{lpar}$), \str{)} ($\textit{rpar}$), and \str{:}. 
Consequently, it partially matches $\regex_\sequence$ as defined by $\pmatch$. 
In Figure~\ref{fig:dfa}, we present the DFAs for $\sequence$ used in computing $\dmatch$. 
The reduction $\dmatch(r.t, \dfastart^\textit{name}, {\textit{lpar}, \textit{rpar}}) = \dmatch(\text{\str{is\_prime():}}, \dfastart^\textit{name}, {\textit{lpar}, \textit{rpar}})$ simplifies successively to $\dmatch(\text{\str{():}}, \dfastart^\textit{lpar}, {\textit{rpar}})$, then to $\dmatch(\text{\str{):}}, \dfastart^\textit{rpar}, {})$, and finally to $\dmatch(\text{\str{:}}, q_1^\textit{rpar}, {})$.
As $q_1^\textit{rpar}$ is a final state, according to condition 2 of Definition~\ref{def:dmatch}, $\dmatch(\text{\str{:}}, q_1^\textit{rpar}, {})$ holds true.
% This equivalence arises as we traverse the $\dfa_\textit{name}$ first, consuming \str{is\_prime} and reaching a final state in $F_\textit{name}$. 
Next, we define a mask over vocabulary

\begin{definition}[Vocabulary mask]
\label{def:mask}
Given vocabulary $\vocab \subseteq \alphabets^*$, $m \in \{0, 1\}^{|\vocab|}$ is a mask over the vocabulary. We also use $\set(m) \subseteq \vocab$ to denote the subset represented by $m$.
\end{definition}

\noindent {\bf DFA Mask Store}
% Discuss \alpha
\noindent For an integer $\alpha$, we define a DFA table $\dmap{\alpha}$ as the mask store over the DFA states with $\alpha$ lookahead. 
Given the set of all DFA states $\dfastates_\Omega = \bigcup_{\terminal \in \allterminals} \dfastates_\terminal$, the table stores binary masks of size $|\vocab|$, indicating for token string $t$, for any DFA state $q \in \dfastates_\Omega$ and a sequence of $\alpha$ terminals $\sequence_\alpha$ if $\dmatch(t, q, \sequence_\alpha) = \true$. 
The lookahead parameter $\alpha$ signifies the number of subsequent terminals considered when generating the mask stored in the table.
Choosing a larger value for $\alpha$ enhances the precision of \Tool{} algorithm, but it comes at the cost of computing and storing a larger table. 
We next formally define the DFA mask store,
\begin{definition}[DFA mask store]
\label{def:lookup}
For an integer $\alpha$, the DFA mask store $\dmap{\alpha}$ is a function defined as $\dmap{\alpha}: \dfastates_\Omega \times \allterminals^{\alpha} \to \{0, 1\}^{|\vocab|}$, where $\dfastates_\Omega = \bigcup_{\terminal \in \allterminals} \dfastates_\terminal$ represents the set of all DFA states and $\allterminals^{\alpha}$ is a set of $\alpha$-length terminal sequences. 
% Let $\textit{id}(t)$ denote the enumerated index of a token $t \in \vocab$. 
Then $\dmap{\alpha}(q, \sequence) = m$ is a binary mask such that $t \in \set(m)$ if $\dmatch(t, q, \sequence)$
\end{definition}

% \begin{wrapfigure}{R}{0.5\textwidth}
\begin{minipage}{0.5\textwidth}
% \vspace{-.55in}

\begin{algorithm}[H] 
\small
\caption{\Tool{} Generation}
\label{alg:main}
% %
\textbf{Inputs:} $M$: LLM, $\tokenizer$: tokenizer, $C_0$: input prompt, $n_\textit{max}$: maximum generated tokens, $D$: decoding strategy
\begin{algorithmic}[1]
\Function{MaskedGenerate}{$M$, $\tokenizer$, $C_0$, $n_\textit{max}$, $D$}
\State $\curtokens \gets \text{Tokenize}(\tokenizer, C_0)$
\For{$i \in \{1, \dots n_\textit{max} \}$}
\State $\textit{scores} \gets M(\curtokens)$
\State $C_k \gets \text{decode}(\tokenizer, \curtokens)$
\State $\accepts, r \gets \text{Parse}(C_k)$
\label{line:acc}
\State $m \gets \text{GrammarMask}(\accepts, r)$
\label{line:gm}
\State $\textit{scores} \gets m \odot \textit{scores}$
\State $t_i \gets D(\textit{scores})$
\If{$t_i = EOS$}
\State break
\EndIf
\State $T_\textit{cur} \gets \text{append}(T_\textit{cur}, t_i)$
% \State $\curtokens \gets \text{append}(\curtokens, t_i)$
\EndFor
\State $\text{output} \gets \text{decode}(\tokenizer, \curtokens)$
\State \Return output
\EndFunction
\end{algorithmic}
\end{algorithm}
\vspace{-0.4in}

\end{minipage}
\end{wrapfigure}

For our illustrative example if $m = \dmap{2}(q_1^\textit{name}, \{\textit{lpar}, \textit{rpar}\})$ then $t=$\str{\_prime():} should be contained in $\set(m)$.
The grammar mask for a set of accept sequences $\accepts$ can be computed by combining masks for each $\sequence \in \accepts$. 
% We describe the grammar mask computation algorithm in Appendix~\ref{sec:compmask}.
The DFA mask store $\dmap{0}$ maps each DFA state to all tokens such that they  $\pmatch$ without considering any following accept sequence (0-length sequence). 
In this case, the table maps each state with a single mask denoting the tokens that match the regular expression of the corresponding DFA. 

\noindent {\bf Computing Grammar Mask}
\label{sec:compmask}
\begin{wrapfigure}{R}{0.53\textwidth}
\vspace{-.34in}
\begin{minipage}{0.55\textwidth}

\begin{algorithm}[H]
\caption{Computing Grammar Mask}
\label{alg:grammar}
%
\textbf{Inputs:} $\accepts$: set of accept sequences, $r$: remainder
\begin{algorithmic}[1]
\Function{GrammarMask}{$\accepts, r$}
\State $m \gets \{\}$
\For{$\sequence \in \accepts$}
\State $\terminal_1 \gets \sequence[0]$
\State $q_r \gets \compute(q_0^{\terminal_1}, r)$
% \For{$q \in \dfastates_{\terminal_1}$}
\If{$q_r \in \live(\dfastates_{\terminal_1})$}
\State $\Pi \gets \textit{len}(\sequence)-1$
\State $m \gets m \cup \big( \dmap{\Pi}(q_r, \sequence[1:])\big)$
\EndIf
% \EndFor
\EndFor
\State \Return $m$
\EndFunction
\end{algorithmic}
\end{algorithm}

\end{minipage}
\vspace{-0.2in}
\end{wrapfigure}
% Example \dmap{0}
% This approach is equivalent to the one used by \cite{willard2023efficient} for regular expression guided generation. 
% The current parsers can easily compute acceptable sequences of terminals with a length of 2 from partial output. 
% We note that $\pmatch$ $r.t$ with a 2-length sequence is equivalent to $\dmatch$ with a 1-length sequence, as stated in Lemma~\ref{lemma:eq}. 
% Consequently, in our work, we opt for $\dmap{0}$ and $\dmap{1}$ since we have observed empirically that this combination is sufficient for producing syntactically valid outputs.
%
The mask store is constructed offline by enumerating all DFA states $\dfastates_\Omega $, considering all possible terminals in $\allterminals$, and all tokens in $\vocab$. 
The DFA mask store depends on the set of terminals $\allterminals$ and the model's vocabulary $\vocab$. 
As a result, a unique mask store is created for each grammar and tokenizer combination, and to enhance efficiency, we cache and reuse this table for future inferences. 

% \noindent{\bf Computing Grammar Mask.}
Algorithm~\ref{alg:grammar} presents our approach for computing the grammar mask during LLM generation.
It computes a grammar mask based on the sets of current accept sequences $\accepts$, and the remainder string ($r$). 
It iterates over $\accepts$, considering each sequence $\sequence$.
The algorithm initializes an empty mask $m$. 
It iterates over each acceptable sequence, considering the first terminal $\terminal_1$ in each. 
It computes the resulting state $q_r$ by processing $\terminal_1$ from an initial state $q_0^{\terminal_1}$ and the remainder string $r$. 
If $q_r$ is in a live state, the algorithm updates the grammar mask by unifying the mask cached in $\dmap{\alpha}$. 