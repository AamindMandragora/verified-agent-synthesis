For the simplicity of presenting the proof, we assume that $d > 2$. 

Since $t \in \set(m)$ for some $\sequence_1 = \{\terminal_{f+1}, \terminal_{f+2} \dots \terminal_{f+d}\} \in \accepts$
\[
    \implies \dmatch(t, q , \{\terminal_{f+2} \dots \terminal_{f+d}\}) \implies \dmatch(r.t, \dfastart^{\terminal_{f+1}}, \{\terminal_{f+2} \dots \terminal_{f+d}\}) 
\]
\[
    \implies \pmatch(r.t, \{\regex_{\terminal_{f+1}}.\regex_{\terminal_{f+2}} \dots \regex_{\terminal_{f+d}}\})
\]
By Definition~\ref{def:pmatch}, there are two possible cases:

\begin{enumerate}
    \item $\exists w_1 \in \alphabets^*, w_2 \in \alphabets^+$ such that $r.t = w_1.w_2 $ and $w_1 \in \lang(\regex_{\terminal_{f+1}}.\regex_{\terminal_{f+2}} \dots \regex_{\terminal_{f+d}})$ \\
    We show that this case is not possible since our terminal sequence $\sequence_1$ is long enough that no prefix of $r.t$ cannot be in $\lang(\regex_{\terminal_{f+1}}.\regex_{\terminal_{f+2}} \dots \regex_{\terminal_{f+d}})$ \\
    We can infer that $\len(w_1) < \len(r.t) \implies \len(w_1) < \len(r) + \len(t)$ \\
    Further, from the assumption $ d > \textit{len}(t)$, we have
    \[
        \len(w_1) < d + \len(r) 
    \]    
    Firstly, note that $r \not\in \lang(\regex_{\terminal_{f+1}}.\regex_{\terminal_{f+2}})$ by the definition of remainder $r$ \\
    Note that we assume no terminal contains empty string i.e. $\epsilon \not \in \lang(\regex_{\terminal_i})$ \\
    Hence, every string in $\lang(\regex_{\terminal_{f+2}} \dots \regex_{\terminal_{f+d}})$ should have length at least $d-1$  \\

    % it's not that clear 
    Clearly, $r$ is prefix of $w_1$. Let $w_3 \in \alphabets^*$, $r.w_3 = w_1$ and hence $\len(w_3) > d-1$ \\
    Hence, 
    \[
        \len(r) + d - 1 < \len(w_1)
    \]
    \[ 
    \len(r) + d - 1 < \len(w_1) < d + \len(r) 
    \]
    This is not possible and hence such $w_1$ cannot exist. 
    
    \item $\exists w_1 \in \alphabets^*$ such that $r.t.w_1 \in \lang(\regex_{\terminal_{f+1}}.\regex_{\terminal_{f+2}} \dots \regex_{\terminal_{f+d}})$

    By Definition~\ref{def:pparse}, we have $\sequence^{\square}, r = \partialparse(\partialcode)$ s.t $\partialcode = \fixpartialcode.r$, $\sequence^{\square} = \terminal_{1}, \terminal_{2} \dots \terminal_{f}$ $\fixpartialcode \in \lang(\regex_{\terminal_{1}}.\regex_{\terminal_{2}} \dots \regex_{\terminal_{f}})$. \\
    Let $\sequence_1 = \terminal_{f+1}, \terminal_{f+2} \dots \terminal_{f+d}$ \\
    Since, $\partialcode. t = \fixpartialcode.r.t$, $\fixpartialcode \in \lang(\sequence^{\square})$ and $r.t.w_1 \in \lang(\sequence_1)$, we have 
    \[
        \fixpartialcode.r.t.w_1 \in \lang(\sequence^{\square}.\sequence_1)
    \]
    \[
        \partialcode.t.w_1 \in \lang(\sequence^{\square}.\sequence_1)
    \]
     By Definition~\ref{def:acc} of accept sequence, $\sequence^{\square} . \sequence_1 \in \lang_p^\allterminals(G)$, Hence
     \[
        \partialcode.t.w_1 \in \lang_p(G) \implies \partialcode.t \in \lang_p(G)
     \]     
\end{enumerate}
Thus, our proof is complete and $\partialcode.t \in \lang_p(G)$