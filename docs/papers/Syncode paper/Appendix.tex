
\appendix
\section{Appendix}

\subsection{List of Symbols}
\label{sec:symbols}
\begin{tabular}{ll}
% Grammar related
    $G$ & Formal Grammar\\
    $L(G)$ & Language of a grammar \\
    $L_p(G)$ & Prefix language of a grammar\\
    $l$ & lexical tokens \\
    $l_i$ & $i$-th lexical token in the parsed output \\
    $\terminal$ & A terminal in the grammar \\
    $\terminal_i$ & Terminal type of $i$-th lexical token \\
    $\allterminals$ & Set of all terminals in the grammar \\
    $\lang^\allterminals(G)$ & Language of terminals for grammar $G$ \\
    $\lang^\allterminals_p(G)$ & Prefix language of terminals \\
    $\parser$ & Parser \\
    $\sequence$ & Sequence of terminals \\
    $\tokenizer$ & Tokenizer in an LLM \\
    $\vocab$ & Vocabulary of an LLM\\
    $\vocab_k$ & Subset of vocabulary containing acceptable tokens at $k$-th LLM generation iteration\\
    $\regex_\terminal$ & Regular expression for a terminal $\terminal$ \\
    $\regex_i$ & Regular expression corresponding to $i$-th lexical token \\
    $\greater$ & Partial order over set of terminal sequences\\

% Syncode related
    $\remainder$ & Remainder from \Tool{} parsing the partial output \\
    $\partialcode$ & Partial output at $k$-th iteration of LLM generation\\
    $\fixpartialcode$ & Parsed prefix of partial output $\partialcode$ at $k$-th iteration of LLM generation\\
    $\accepts$ & Set of accept sequences \\
    $\dmap{\alpha}$ & DFA lookup store function for terminal sequences of length $\alpha$ \\
    $\dmatch$ &  Match with DFA walk as defined in Section~\ref{sec:technical}\\
    $\pmatch$  & Partial match with regular expression\\
    $\partialparse$ & Partial parsing function \\
    $m$ & Boolean mask \\



% DFA related
    $\dfa$ & Deterministic finite automaton \\
    $\dfastates$ & States in a DFA \\
    $\alphabets$ & Set of characters i.e. alphabet for DFA \\
    $\transitions$ & Transition function in a DFA \\
    $\compute$ &  Extended transition function in a DFA\\
    $\dfastart$ & Start state of a DFA \\
    $\dfafinal$ & Set of final states in DFA\\
    $\live$ & Live states of the DFA \\
    $\dfastates_\Omega$ & Set containing all DFA states for DFAs of all terminals in the grammar\\

% Parser related 
    $\curaccepts$ & Set of terminals acceptable for current lexical token \\
    $\nextaccepts$ & Set of terminals acceptable as for next lexical token \\
    $\lex$ & Lexer function \\
    $\len$ & Length of a sequence \\
    $\curtokens$ & Current set of tokens \\
    $\psmap$ & Map for storing parser state \\
\end{tabular}


% \newcommand{\partialparse}{\textit{pparse}}

% \newcommand{\set}{\textit{set}}

% \newcommand{\str}[1]{\enquote*{\emph{#1}}} 

% % Parser related
% \newcommand{\next}{\emph{Next}\xspace}
% \newcommand{\follow}{\emph{Follow}\xspace}
% % \newcommand{\lexertokens}{\textit{LT}}

% % regex
% \newcommand{\pmatch}{\textit{pmatch}}
% \newcommand{\len}{\textit{len}}

% % DFA related
% \newcommand{\dmatch}{\textit{dmatch}}


\newpage
\subsection{Proofs for Theorems}
\label{sec:proofs}

\eq* 
\begin{proof}
\begin{enumerate} [label=(\alph*)]
    \item First we prove $\dmatch(w, \dfastart^{\terminal_{f+1}}, \sequence^p) \implies \pmatch(w, \regex_\sequence)$
    We prove this using induction on the length $i$ of $w$. \\
    For $i=0$, $\pmatch(w, \regex_\sequence)$ is trivially true. \\
    Now, we assume that for $w$ of length $i < k$, $\dmatch(w, \dfastart^{\terminal_{f+1}}, \sequence^p) \implies \pmatch(w, \regex_\sequence)$. \\
    We consider $w$ of length $k$ and $\dmatch(w, \dfastart^{\terminal_{f+1}}, \sequence^p)$.\\ 
    We consider 3 conditions from Definition~\ref{def:dmatch}. \\
    If condition 1 is true, $\compute_{\terminal_{f+1}}(w, \dfastart^{\terminal_{f+1}}) \in \live(Q_{\terminal_{f+1}})$. 
    Let $q_1 = \compute(w, \dfastart^{\terminal_{f+1}})$. 
    By Definition~\ref{def:live}, $\exists w_1 \text{ s.t. } \compute_{\terminal_{f+1}}(w_1, q_1) \in F_{\terminal_{f+1}}$.
    Hence, 
    \[\compute(w.w_1, \dfastart^{\terminal_{f+1}}) \in F_{\terminal_{f+1}} \implies w.w_1 \in \lang(\regex_{\terminal_{f+1}})\]
    We assume that each terminal $\lang(\terminal_i)$ is non-empty. Hence, 
    \[\exists w_2 \in \lang(\regex_{\sequence^p}) \implies w.w_1.w_2 \in \lang(\regex_{\sequence}) \]
    Hence, by condition 2 from Definition~\ref{def:pmatch}, $\pmatch(w, \regex_\sequence)$. 
    
    If condition 2 is true, $\exists w_1, w_2$ such that $w_1.w_2 = w$, $\compute_{\terminal_{f+1}}(w_1, \dfastart^{\terminal_{f+1}}) \in F \text{ and } \sequence^p= \{\} $. 
    Here, $w_1 \in \lang(\regex_{\terminal_{f+1}})$. 
    Since $\sequence^p = \{\}$, $\regex_\sequence = \regex_1$, and hence, $w_1 \in \lang(\regex_\sequence)$. 
    Hence by condition 1 from Definition~\ref{def:pmatch}, $\pmatch(w, \regex_\sequence)$. 

    If condition 3 is true, $\exists w_1, w_2$ such that $w_1.w_2 = w$, $\compute_{\terminal_{f+1}}(w_1, \dfastart^{\terminal_{f+1}}) \in F_{\terminal_{f+1}}$, \\ 
    and $\dmatch(w_2, q_{0}^{\terminal_{f+2}}, \{\terminal_{f+3} \dots \terminal_{f+d}\}) = \true$. 
    \[\compute_{\terminal_{f+1}}(w_1, \dfastart^{\terminal_{f+1}}) \in F_{\terminal_{f+1}} \implies w_1 \in \lang(\regex_{\terminal_{f+1}}) \]
    Since length of $w_2 < k$, by our induction hypothesis, $\pmatch(w_2, \regex_{\sequence^p}) = \true.$ By Definition~\ref{def:pmatch}, there are two possibilities. 
    Suppose $\exists w_2 = w_3.w_4$ such that $w_3 \in \lang(\regex_{\sequence^p})$. \\
    \[w_1.w_3 \in \lang(\regex_{\sequence}) \implies \pmatch(w, \regex_{\sequence}) = \true \] 
    Alternatively, if $\exists w_3$ such that $w_2.w_3 \in \lang(\regex_{\sequence^p})$ 
    \[ w_1.w_2.w_3 \in \lang(\regex_\sequence) \implies \pmatch(w, \regex_{\sequence}) = \true
    \]
    Hence, our induction proof is complete and $\pmatch(w, \regex_{\sequence}) = \true$


    \item Next we prove $\pmatch(w, \regex_\sequence) \implies \dmatch(w, \dfastart^{\terminal_{f+1}}, \sequence^p)$
    We prove this using induction on the length $i$ of $w$. \\
    For $i=0$, $\dmatch(w, \dfastart^{\terminal_{f+1}}, \sequence^p)$ is trivially true. \\
    Now, we assume that for $w$ of length $i < k$, $ \pmatch(w, \regex_\sequence) \implies \dmatch(w, \dfastart^{\terminal_{f+1}}, \sequence^p)$ \\
    Now we consider $w$ of length $k$ and $\pmatch(w, \regex_\sequence)$.\\ 
    By Definition~\ref{def:pmatch}, there are two possible conditions \\
    
    \textbf{Case 1:} $\exists w_1 \in \alphabets^*, w_2 \in \alphabets^+\text{ such that } w = w_1.w_2$ and $w_1 \in \lang(\regex_\sequence)$ \\
    Hence, $\exists w_3, w_4 \text{ such that } w_1 = w_3.w_4 $ and $w_3 \in \lang(\regex_{\terminal_{f+1}})$ and $w_4 \in \lang(\regex_{\sequence^p})$. By induction hypothesis, 
    \[
        \pmatch(w_4.w_2, \regex_{\sequence^p}) \implies \dmatch(w_4 w_2, \{\terminal_{f+2}, \terminal_{f+3} \dots \terminal_{f+d} \})
    \]
    Since $w = w_3.w_4.w_2$ and
    \[
        w_3 \in \lang(\regex_{\terminal_{f+1}}) \implies \compute_{\terminal_{f+1}}(w_3, \dfastart^{\terminal_{f+1}}) \in F_{\terminal_{f+1}}
    \]
    Hence, by condition 3 in Definition~\ref{def:dmatch}, $\dmatch(w, \dfastart^{\terminal_{f+1}}, \sequence^p)$  

    \textbf{Case 2:} $\exists w_1$ such that $w.w_1 \in \lang(\regex_\sequence)$ \\
    Hence, $\exists w_2, w_3 \text{ s.t } w.w_1 = w_2.w_3 $ and $w_2 \in \lang(\regex_{\terminal_{f+1}})$ and $w_3 \in \lang(\regex_{\sequence})$ \\
    Now there are two possibilities, either $w$ is prefix of $w_2$ or $w_2$ is prefix of $w_2$ \\
    Supoose $w$ is prefix of $w_2$, then $\compute_{\terminal_{f+1}}(w, \dfastart^{\terminal_{f+1}}) \in \live(\dfastates_{\terminal_{f+1}})$ and hence by Definition~\ref{def:dmatch}, $\dmatch(w, \dfastart^{\terminal_{f+1}}, \sequence^p)$
    Alternatively, if $w_2$ is prefix of $w$ then $\exists w_4 \text{ s.t. } w = w_2 w_4$ \\
    Hence, $w_4.w_1 = w_3 \in \lang(\regex_{\terminal_{f+1}})$ and thus $\pmatch(w_4, \regex_{\sequence^p})$ \\
    By induction hypothesis $\dmatch(w_4, \dfastart^{\terminal_{f+2}}, \{\terminal_{f+3}, \terminal_4 \dots \terminal_{f+d} \})$ \\
    and since $w = w_2.w_4 $ and $\compute_{\terminal_{f+1}}(w_2, \dfastart^{\terminal_{f+1}}) \in F_{\terminal_{f+1}}$. 
    We get $\dmatch(w, \dfastart^{\terminal_{f+1}}, \sequence^p)$
    
\end{enumerate}


\end{proof}

\dm*
\begin{proof}
\begin{enumerate}[label=(\alph*)]
    \item First, we prove $\dmatch(t, q, \sequence) \implies \dmatch(r.t, \dfastart^{\terminal}, \sequence)$. \\
    From Definition~\ref{def:dmatch}, either of the 3 conditions hold true for $\dmatch(t, q, \sequence)$. \\
    
    If condition 1 is true then 
\[\compute_{\terminal_1}(t, q) \in \live(Q_{\terminal}) \implies \compute_{\terminal}(r.t, \dfastart^{\terminal}) \in \live(Q_{\terminal}) \implies \dmatch(r.t, \dfastart^{\terminal}, \sequence)\]

    If condition 2 is true,  $\exists w_1, w_2$ such that $w_1.w_2 = t$, $\compute_{\terminal}(w_1, q) \in F \text{ and } \sequence= \{\}$. Therefore,  
\[\compute_{\terminal}(r.w_1, q) \in F \implies \dmatch(r.t, \dfastart^{\terminal}, \sequence)\]

    If condition 3 is true, $\exists w_1, w_2$ such that $w_1.w_2 = t$, $\compute_{\terminal}(w_1, q) \in F$ \\ and $\dmatch(w_2, q_{0}^{\terminal_1}, \{\terminal_2 \dots \terminal_d\}) = \true$. Therefore, \[\compute_{\terminal}(r.w_1, q) \in F \implies \dmatch(r.t, \dfastart^{\terminal}, \sequence)\]

Therefore, in all cases, $\dmatch(rt, \dfastart^{\terminal}, \sequence)$ must hold. 

    \item Now, we prove $\dmatch(rt, \dfastart^{\terminal}, \sequence)  \implies \dmatch(t, q, \sequence)$.

     From Definition~\ref{def:dmatch}, either of the 3 conditions hold true for $\dmatch(r.t, \dfastart^{\terminal}, \sequence)$. \\

     If condition 1 is true then 
    \[
    \compute_{\terminal_1}(r.t, \dfastart^{\terminal}) \in \live(Q_{\terminal}) \implies \compute_{\terminal}(t, q) \in \live(Q_{\terminal}) \implies \dmatch(t, q, \sequence)
    \]

    If condition 2 is true,  $\exists w_1, w_2$ such that $w_1.w_2 = r.t$, $\compute_{\terminal}(w_1, \dfastart^{\terminal}) \in F \text{ and } \sequence= \{\}$. Since no prefix of $r$ is accepted by $\lang(\terminal)$, $\exists w_3 \text{ s.t. } w_3 w_4 = t$ and
    \[
    \compute_{\terminal}(w_3, q) \in F \implies \dmatch(t, q, \sequence)
    \]

    If condition 3 is true, $\exists w_1, w_2$ such that $w_1.w_2 = r.t$, $\compute_{\terminal}(w_1, \dfastart^{\terminal}) \in F$ \\ 
    and $\dmatch(w_2, q_{0}^{\terminal_1}, \{\terminal_2 \dots \terminal_d\}) = \true$. Since no prefix of $r$ is accepted by $\lang(\terminal)$, $\exists w_3 \text{ s.t. } w_3 w_4 = t$ and  
    \[\compute_{\terminal}(w_3, q) \in F \implies \dmatch(t, q, \sequence)\]

    Therefore, in all cases, $\dmatch(t, q, \sequence)$ must hold. 
    
\end{enumerate}

\end{proof}

\Sound*
\begin{proof}
% Since $\partialcode.t \in \lang_p(G)$, $\exists w \text{ s.t. } \partialcode.t.w \in \lang(G)$. \\
Let $r, \sequence^{\square} = \partialparse(\partialcode)$ where $\sequence^{\square} = \terminal_1, \terminal_2 \dots \terminal_{f}$ and let $r_1, \sequence_1 = \partialparse(\partialcode.t)$ where $\sequence_1 = \terminal_1, \terminal_2 \dots \terminal_{f} \dots \terminal_{f+g}$ \\
Hence, we can split $r.t$ such that for $w \in \alphabets^*$, $r.t = w.r_1$ and $w \in \lang(\terminal_{f+1} \dots \terminal_{f+g})$ \\
There are two possible cases: \\
\noindent\textbf{Case 1:} $g < d$ \\
\[
    w \in \lang(\terminal_{f+1} \dots \terminal_{f+g})
\]
\[
    \implies w \in \lang_p(\terminal_{f+1} \dots \terminal_{f+g})
\]
By our assumption on $\accepts_d$ there must exist $\sequence_2 = \terminal_{f+1} \dots \terminal_{f+d}$ s.t. $\terminal_{f+1} \dots \terminal_{f+g}$ is prefix of $\sequence_2$. Hence, 
\[
    \implies w \in \lang_p(\sequence_2)
\]
\[
    \implies \pmatch(r.t, \sequence_2)
\]


\noindent\textbf{Case 2:} $g \geq d$ \\
Since we assume that $\accepts_d$ contains all possible accept sequence of length $d$, $\sequence_2 = \terminal_{f+1} \dots \terminal_{f+d}$ must be contained in $\accepts_d$ \\
Hence, $\exists w_1, w_2 \in \alphabets^*$ such that $w = w_1.w_2$ and
\[
    w_1 \in \lang(\sequence_2)
\]
\[
    \implies w \in \lang_p(\sequence_2)
\]
\[
    \implies \pmatch(r.t, \sequence_2)
\]
% 
In both cases, $\pmatch(r.t, \sequence_2)$. Using Lemma~\ref{lemma:eq},
\[
    \implies \dmatch(r.t, \dfastart^{\terminal_{f+1}} , \{\terminal_{f+2} \dots \terminal_{f+d}\}) 
\]
Using Lemma~\ref{lemma:dmatch} if $q = \compute_{\terminal_{f+1}}(r, \dfastart^{\terminal_{f+1}})$
\[
    \dmatch(r.t, \dfastart^{\terminal_{f+1}} ,\{\terminal_{f+2} \dots \terminal_{f+d}\}) \implies \dmatch(t, q, \{\terminal_{f+2} \dots \terminal_{f+d}\})
\]
Here from Definition~\ref{def:lookup}, if $\dmap{d-1}(q, \{\terminal_{f+2} \dots \terminal_{f+d}\}) = m_2$ then $t \in \set(m_2)$. \\
Since $m_2 \subseteq m$, we have our result $t \in \set(m)$.

% and let $\sequence_1 = \terminal_{f+1}, \terminal_{f+2} \dots \terminal_{f+d} \in \accepts_d$\\


% We consider 2 cases from Algorithm~\ref{alg:parse}. \\
% \noindent\textbf{Case 1:} $r \in \lang(\regex_{\terminal_{f+1}})$ \\
% Let $\sequence = \terminal_{f+1} \terminal_{f+2}$. Clearly, $\sequence \in \accepts$ returned by Algorithm~\ref{alg:parse} . \\
% Since $\terminal_i, \terminal_{f+1} \dots \terminal_{q}$ denote the terminals for $r.t.w_1$, there are two possibilities. 
% Either $\exists w_2 \text{ s.t. } r.t.w_2 \in \lang(\terminal_i.\terminal_{f+1})$ or $\exists w_3 \text{ s.t. } r.t = w_3.w_4 \text{ and } w_3 \in \lang(\terminal_i.\terminal_{i+1})$. \\
% In both the cases, $\pmatch(rt, \{\terminal_i, \terminal_{i+1}\})$. Using Lemma~\ref{lemma:eq},
% \[
%     \pmatch(r.t, \{\terminal_i, \terminal_{i+1}\}) \implies \dmatch(rt, \dfastart^{\terminal_i} ,\{\terminal_{i+1}\}) 
% \]
% Using Lemma~\ref{lemma:dmatch} if $q = \compute_{\terminal_i}(r, \dfastart^{\terminal_i})$
% \[
%     \dmatch(r.t, \dfastart^{\terminal_i} ,\{\terminal_{i+1}\}) \implies \dmatch(t, q ,\{\terminal_{i+1}\})
% \]
% Here from Definition~\ref{def:lookup}, if $\dmap{1}(q, \{\terminal_{i+1}\}) = m_1$ then $m_1[t] = 1$. \\
% Since $m$ includes $m_1$, we have our result $m[t] = 1$. \\

% \noindent\textbf{Case 2:} $r \not\in \lang(\regex_{\terminal_{i}})$
% Let $\sequence = \terminal_i$. Clearly, $\sequence \in \accepts$ returned by Algorithm~\ref{alg:parse} . \\
% Since $\terminal_i, \terminal_{i+1} \dots \terminal_{q}$ denote the terminals for $r.t.w_1$, there are two possibilities. 
% Either $\exists w_2 \text{ s.t. } r.t.w_2 \in \lang(\terminal_i)$ or $\exists w_3 \text{ s.t. } r.t = w_3.w_4 \text{ and } w_3 \in \lang(\terminal_i)$. \\
% In both the cases, $\pmatch(r.t, \{\terminal_i\})$. Using Lemma~\ref{lemma:eq},
% \[
%     \pmatch(r.t, \{\terminal_i\}) \implies \dmatch(r.t, \dfastart^{\terminal_i} ,\{\}) 
% \]
% Using Lemma~\ref{lemma:dmatch} if $q = \compute_{\terminal_i}(r, \dfastart^{\terminal_i})$
% \[
%     \dmatch(r.t, \dfastart^{\terminal_i} ,\{\}) \implies \dmatch(t, q ,\{\})
% \]
% Here from Definition~\ref{def:lookup}, if $\dmap{0}(q, \{\}) = m_2$ then $m_2[t] = 1$. \\
% Since $m$ includes $m_2$, we have our result $m[t] = 1$.
\end{proof}

\Porder*
\begin{proof}
Since $\forall \sequence_2 \in \accepts_2 \exists \sequence_1 \in \accepts_1 \exists \sequence_3 \in \allterminals^*  $  
s.t. $\sequence_2 = \sequence_1.\sequence_3$, Hence 
\[
\pmatch(w, \regex_{\sequence_2}) \implies \pmatch(w, \regex_{\sequence_1})
\]
Hence, for the mask $\set(m_2) \subseteq \set(m_1)$
\end{proof}

\Complete*
For the simplicity of presenting the proof, we assume that $d > 2$. 

Since $t \in \set(m)$ for some $\sequence_1 = \{\terminal_{f+1}, \terminal_{f+2} \dots \terminal_{f+d}\} \in \accepts$
\[
    \implies \dmatch(t, q , \{\terminal_{f+2} \dots \terminal_{f+d}\}) \implies \dmatch(r.t, \dfastart^{\terminal_{f+1}}, \{\terminal_{f+2} \dots \terminal_{f+d}\}) 
\]
\[
    \implies \pmatch(r.t, \{\regex_{\terminal_{f+1}}.\regex_{\terminal_{f+2}} \dots \regex_{\terminal_{f+d}}\})
\]
By Definition~\ref{def:pmatch}, there are two possible cases:

\begin{enumerate}
    \item $\exists w_1 \in \alphabets^*, w_2 \in \alphabets^+$ such that $r.t = w_1.w_2 $ and $w_1 \in \lang(\regex_{\terminal_{f+1}}.\regex_{\terminal_{f+2}} \dots \regex_{\terminal_{f+d}})$ \\
    We show that this case is not possible since our terminal sequence $\sequence_1$ is long enough that no prefix of $r.t$ cannot be in $\lang(\regex_{\terminal_{f+1}}.\regex_{\terminal_{f+2}} \dots \regex_{\terminal_{f+d}})$ \\
    We can infer that $\len(w_1) < \len(r.t) \implies \len(w_1) < \len(r) + \len(t)$ \\
    Further, from the assumption $ d > \textit{len}(t)$, we have
    \[
        \len(w_1) < d + \len(r) 
    \]    
    Firstly, note that $r \not\in \lang(\regex_{\terminal_{f+1}}.\regex_{\terminal_{f+2}})$ by the definition of remainder $r$ \\
    Note that we assume no terminal contains empty string i.e. $\epsilon \not \in \lang(\regex_{\terminal_i})$ \\
    Hence, every string in $\lang(\regex_{\terminal_{f+2}} \dots \regex_{\terminal_{f+d}})$ should have length at least $d-1$  \\

    % it's not that clear 
    Clearly, $r$ is prefix of $w_1$. Let $w_3 \in \alphabets^*$, $r.w_3 = w_1$ and hence $\len(w_3) > d-1$ \\
    Hence, 
    \[
        \len(r) + d - 1 < \len(w_1)
    \]
    \[ 
    \len(r) + d - 1 < \len(w_1) < d + \len(r) 
    \]
    This is not possible and hence such $w_1$ cannot exist. 
    
    \item $\exists w_1 \in \alphabets^*$ such that $r.t.w_1 \in \lang(\regex_{\terminal_{f+1}}.\regex_{\terminal_{f+2}} \dots \regex_{\terminal_{f+d}})$

    By Definition~\ref{def:pparse}, we have $\sequence^{\square}, r = \partialparse(\partialcode)$ s.t $\partialcode = \fixpartialcode.r$, $\sequence^{\square} = \terminal_{1}, \terminal_{2} \dots \terminal_{f}$ $\fixpartialcode \in \lang(\regex_{\terminal_{1}}.\regex_{\terminal_{2}} \dots \regex_{\terminal_{f}})$. \\
    Let $\sequence_1 = \terminal_{f+1}, \terminal_{f+2} \dots \terminal_{f+d}$ \\
    Since, $\partialcode. t = \fixpartialcode.r.t$, $\fixpartialcode \in \lang(\sequence^{\square})$ and $r.t.w_1 \in \lang(\sequence_1)$, we have 
    \[
        \fixpartialcode.r.t.w_1 \in \lang(\sequence^{\square}.\sequence_1)
    \]
    \[
        \partialcode.t.w_1 \in \lang(\sequence^{\square}.\sequence_1)
    \]
     By Definition~\ref{def:acc} of accept sequence, $\sequence^{\square} . \sequence_1 \in \lang_p^\allterminals(G)$, Hence
     \[
        \partialcode.t.w_1 \in \lang_p(G) \implies \partialcode.t \in \lang_p(G)
     \]     
\end{enumerate}
Thus, our proof is complete and $\partialcode.t \in \lang_p(G)$

\newpage
\subsection{Incremental Parsing Algorithm}
\label{sec:incparse}
\begin{wrapfigure}{R}{0.53\textwidth}
\vspace{-.2in}
\begin{minipage}{0.55\textwidth}

\begin{algorithm}[H]
\caption{Incremental Parsing}
\label{alg:parse}
%
\textbf{Inputs:} $\partialcode$: partial output, $\psmap$: state map
\begin{algorithmic}[1]
\Function{Parse}{$\partialcode$}
\State $l_1, l_2 \dots l_f \gets \text{\lex}(\partialcode)$
% \State $j \gets \text{Len}(L)$
\State $\gamma, S_{\gamma} \gets \text{RestoreState}(\psmap, L)$
\State $\parser \gets \text{Initialize}(S_\gamma)$
\State $\textit{parsed} \gets l_1.l_2\dots l_{\gamma-1}$
\For{$l_i \in l_\gamma, l_{\gamma+1} \dots l_f$}
\State $\text{\next}(\parser, l_i)$
\If{$\parser.state = Error$}
\State break
\EndIf
\State $\textit{parsed} \gets \textit{parsed} + l_i$
\State $\curaccepts \gets \nextaccepts$
\State $\nextaccepts \gets \text{\follow}(\parser)$
\State $S_i \gets \parser.state$
\State $\text{Store}(\psmap, \textit{parsed}, S_i)$
\EndFor
% \State $\text{output} \gets \text{decode}(T, \curtokens)$
% \State $r \gets \partialcode[\textit{parsed}:]$
\If{$\partialcode = \textit{parsed}$}
\State $r = l_f$
\State $\accepts \gets \{\terminal_f, \nextaccepts[0]\}, \{\terminal_f, \nextaccepts[1]\} \dots \}$
\State \qquad $\cup \{\curaccepts[0]\}, \{\curaccepts[1]\} \dots \}$
\Else
\State $r = \partialcode - \textit{parsed}$
\State $\accepts \gets \{\nextaccepts[0]\}, \{\nextaccepts[1]\} \dots $
\EndIf

\State \Return $\accepts, r$
\EndFunction
%\item[]
\end{algorithmic}
\end{algorithm}

\end{minipage}
\vspace{-0.2in}
\end{wrapfigure}
Our parsing algorithm achieves incrementality in LLM generation by utilizing a map $\psmap$ to store the parser state. 
This map associates a list of lexical tokens with the corresponding parser state after parsing those tokens. 
Frequently, in subsequent LLM generation iterations, the count of lexical tokens remains the same—either the next vocabulary token is appended to the final lexical token, or it increases. 
Although uncommon, there are cases where the number of parsed lexical tokens may decrease during iterations.
For example, in Python, an empty pair of double quotes, "", is recognized as a complete lexical token representing an empty string. 
On the other hand, """ serves as a prefix to a docstring, constituting an incomplete parser token. 
Consequently, the addition of a single double quote " reduces the overall count of lexical tokens in these iterations.
We observe that while the total count of lexer tokens at the end may undergo slight changes during these iterations, the majority of prefixes of the parsed lexical tokens remain consistent. 
Hence, we establish a mapping between lists of prefixes of lexical tokens and the corresponding parser state after parsing those tokens. 
Subsequently, when parsing a new list of lexer tokens, we efficiently determine the maximum length prefix of the lexer token list that is already present in $\psmap$. 
This incremental approach significantly reduces the complexity of our parsing algorithm. 
% The pseudocode for our parsing algorithm is presented in Appendix~\ref{sec:incparsealgo}.

% Lexing not incremental
While it could be feasible to introduce incrementality in the lexing operation, our experiments revealed that lexing consumes insignificant time in comparison to parsing. 
As a result, we opted to focus only on performing parsing incrementally.

% base parser
Our incremental parsing algorithm uses a standard non-incremental base parser $\parser$ that maintains a parser state and supports two functions \next and \follow. 
The \next function accepts the next lexer token and then updates the parser state. 
The \follow function returns a list of acceptable terminals at the current parser state. 
These functions are present in common parser generator tools \cite{lark,antlr}.

% Algorithm description
The Algorithm~\ref{alg:parse} presents our incremental parsing algorithm. 
The algorithm utilizes a lexer to tokenize the partial output.
The function RestoreState is used to restore the state of the parser to the maximal matching prefix of lexical tokens that exist in $\psmap$.
The main loop iterates through the tokens and maintains a parser state map. 
For each token, it updates the parser state, stores the mapping in $\psmap$, and retrieves the next set of acceptable terminals. 
The process continues until the end of the partial output. 
The algorithm returns accept sequences $\accepts$ and remainder $r$.

% \subsection{Handling Indentation in Python}
% \add{
% While the \Tool{} framework effectively decodes most of Python's syntax without additional handling, the language's indentation poses a challenge, as it cannot be expressed by CFGs. 
% In \Tool{}, we offer users a seamless interface to extend the existing framework with additional syntactic rules. 
% For Python, we guarantee the syntactic validity of the LLM-generated code by calculating valid indentation levels for the next token and generating a token mask based on indentation constraints.
% }



% \subsection{Evaluation Models}
% \label{sec:models}
% % \begin{itemize}
% % \itemsep0em 
% \noindent \textbf{CodeGen-350M-multi} ~\cite{nijkamp2023codegen}. It is a member of the CodeGen series.
% With its 28-layer architecture, it has 350M parameters, a hidden state size of 4096, 16 attention heads, and a diverse vocabulary of 50400 tokens. 
% It is pre-trained on the BigQuery dataset~\cite{nijkamp2023codegen}, which encompasses open-source code from six programming languages, including C, C++, Java, JavaScript, Go, and Python.

% \noindent \textbf{WizardCoder-1B}~\cite{luo2023wizardcoder}. It is fine-tuned from StarCoder\cite{li2023starcoder}. 
% It employs the Evol-Instruct~\cite{xu2023wizardlm} methodology for refining the training dataset, ensuring simpler and more consistent prompts. 
% The model features 24 transformer layers, 2048-dimensional hidden states, 16 attention heads, over 1B parameters, and a vocabulary count of 49153.

% \noindent \textbf{\llama{}}~\cite{touvron2023llama}. It is from the Llama model family and engineered for advanced natural language processing tasks, including code synthesis. The model is structured with 32 transformer layers, 4096-dimensional hidden states, 32 attention heads, 7 billion parameters, and a diverse vocabulary of 32000 tokens. Its pre-training regime encompasses a diverse set of data sources such as CommonCrawl, C4, Github, Wikipedia, Books, ArXiv, and StackExchange.
% % \end{itemize}




\subsection{Ablation Studies}
In this section, we perform an ablation study for incremental parsing and max new tokens. 
% In Appendix~\ref{sec:lalr_ablation} we perform ablations comparing LALR and LR parsers as a base parser.

\label{sec:ablation}
\noindent{\bf Incremental Parsing.}
We compare the runtime efficiency of utilizing incremental parsing over re-running parsing from scratch in \Tool{}. 
We run inference on \codegen{} with \Tool{} using incremental parsing and parsing from scratch on Python prompts from the HumanEval dataset. 
We generate $n = 1$ samples and control the max new tokens in the code completion. 
Our results are presented in Figure~\ref{fig:inc_parser}, where the x-axis represents the max new tokens and the y-axis represents the average generation time, in seconds, with and without incremental parsing. 
As shown in the figure, the average generation time when re-parsing from scratch increases significantly as the maximum length of code that the LLM can generate increases. 
On the other hand, the average generation time increases slowly with incremental parsing. 
For max new tokens = 300, \Tool{} with incremental parsing achieves 9x speedup over running parsing from scratch. 
Our results collectively demonstrate that augmenting \Tool{} with incremental parsing dramatically improves generation time, especially when generating longer completions.
% \TBD{Increase the font of the labels in Fig 10! }
%---------------------------------------------------------------%
\begin{figure}[!htbp]
\centering
\vspace{-.1in}
\begin{subfigure}[b]{0.48\textwidth}
    \includegraphics[width=1.10\textwidth]{images/SynCode_vs_Standard.png}
    \caption{Average generation time for different max new tokens}
    \label{fig:SynCode_vs_Standard}
\end{subfigure}
\hspace{3mm}
\begin{subfigure}[b]{0.48\textwidth}
    \includegraphics[width=1.10\textwidth]{images/inc_speed_up.png}
    \caption{Average generation time with and without incremental parser}
    \label{fig:inc_parser}
\end{subfigure}
 \vspace{-.1in}
\hfill
\caption{Ablation studies on \codegen{} model.}
\vspace{-0.05in}
\end{figure} 

%---------------------------------------------------------------%
\noindent{\bf Max New Tokens.} 
%
We conduct an ablation study into the relationship between the maximum length of code that the LLMs can generate and generation times. 
We used Python prompts from the HumanEval dataset and leveraged \codegen{} to generate the code completions, both with and without the augmentation of the \Tool{}.
%We compute the average generation time, which is measured by dividing the total generation time for the dataset by the total number of problems. 
As shown in Figure~\ref{fig:SynCode_vs_Standard}, as we increase the max new tokens, we observe a corresponding increase in generation time. 

% \begin{table}[h]
\centering
\tablesize
% \vspace{-0.32in}

\caption{Avg. time taken for a single prompt generation with LR(1) and LALR(1)}
\begin{tabular}{@{}lrrrr@{}}
    \toprule
    Model & \multicolumn{2}{c}{Python(s)} & \multicolumn{2}{c}{Go(s)} \\
    
    & LALR(1) & LR(1) & LALR(1) & LR(1) \\
    \midrule
    CodeGen-350M & 6.06 & \textbf{4.88} & 6.79 & \textbf{5.12} \\
    WizardCoder-1B & 3.00 & \textbf{1.74} & 3.69 & \textbf{1.99} \\
    \llama{} & 7.26  & \textbf{6.07} & 7.97 & \textbf{6.67} \\
    
    \bottomrule
\end{tabular}
\vspace{-0.1in}

\label{table:parsers_runtime}
\end{table}

    % CodeGen-350M & 6.06 & 6.79 & 4.88 & 5.12 \\
    % WizardCoder-1B & 3.00 & 3.69 & 1.74 &  1.99 \\
    % Llama-7B & 7.26  & 7.97  & 6.07 & 6.67 \\

% \begin{table}
% \centering
% \tablesize
% % \vspace{-0.32in}

% \caption{Avg. time taken for a single prompt generation}
% \begin{tabular}{lllll}
%     \toprule
%     Model & Language & \Baseline{}(s) & \multicolumn{2}{c}{\Tool{}(s)} \\
    
%     & & & LALR(1) & LR(1) \\
%     \midrule
%     CodeGen-350M & Python & 4.36  & 6.06 & 4.88\\
%     & Go & 4.35  & 6.79 & 5.12\\
%     \hline
%     WizardCoder-1B & Python & 1.49  & 3.00 & 1.74\\
%     & Go & 1.42  & 3.69 & 1.99\\
%     \hline
%     Llama-7B & Python & 4.88  & 7.26 & 6.07\\
%     & Go & 5.35  & 7.97 & 6.67\\
    
%     \bottomrule
% \end{tabular}
% \vspace{-0.1in}

% \label{table:runtime}
% \end{table}
% \noindent{\bf LR(1) and LALR(1).} 
% % \subsection{LR(1) and LALR(1)}
% \label{sec:lalr_ablation}
% We compare the runtime efficiency of utilizing LR(1) and LALR(1) parsing in \Tool{}.
% We run inference on \codegen{}, \wizard{}, and \llama{} with \Tool{} with LALR(1) parser and with LR(1) parser for Python and Go on the HumanEval dataset. 
% We generate a single sample $(n = 1)$ per prompt with the max new tokens parameter set to 200. 
% Table~\ref{table:parsers_runtime} reports the average time taken to generate each prompt from the datasets. 
% As shown in Table~\ref{table:parsers_runtime}, we observe that \Tool{} with LR(1) parser outperforms the LALR(1) parser with respective overheads of 1.22x on average and 1.76x on average compared to the \baseline{} generation result from~\ref{table:runtime}.

\subsection{Few-Shot Prompting}
\label{sec:few_shot}
\begin{table}
    \tablesize
    \centering
    \vspace{-0.1in}
    \caption{\Tool{} on few-shot prompting}
    \begin{tabular}{@{}ll ccc@{}}
        \toprule
        Architecture & Error Type & \Baseline{} & \Tool{} & $\downarrow$ \\
        \hline
         CodeGen-350M  & Syntax & 53 & 0 & 100\% \\
          & Indentation & 15 & 3 & 80\% \\
        WizardCoder-1B  & Syntax & 40 & 2 & 95\% \\
        & Indentation & 22 & 1 & 95\% \\
        Llama-7B & Syntax & 110 & 0 & 100\% \\
        & Indentation & 40 & 5 & 88\% \\
        
        \bottomrule
    \end{tabular}
    \vspace{-0.1in}
    \label{tab:few_shot}
\end{table}
Few-shot prompting \cite{ren2018metalearning} refers to the idea that language models do not need to be specifically
trained for a downstream task such as classification or question answering. Rather, it is sufficient to train them on broad text-sequence prediction datasets and to provide context in the form of examples when invoking them. We study the performance of utilizing \Tool{} on few-shot prompting code generation tasks. We selected Python few-shot examples from the MBXP dataset and generated code completions with \codegen{}, \llama{}, and \wizard{} with \Tool{} and the \baseline{} no-masking generation. We present our results in Table \ref{tab:few_shot}. The columns standard and SynCode represent the total number of errors of a particular Error Type of LLM-generated code completions to problems in a particular dataset when utilizing that respective generation approach. The column ↓ represents the percentage reduction from the standard column to the SynCode column. As shown in the table, \Tool{} exhibits effectiveness not only in zero-shot but also in the context of few-shot prompting tasks. This signifies the versatility of \Tool{} in enhancing code generation across different prompt engineering techniques.

\subsection{\Tool{} Syntax Errors}
\label{sec:syncode_error}
\begin{figure}[tbh]
    \centering
    \includegraphics[width=\textwidth]{images/syncode_fail.png}
    \caption{Syntactically Incorrect \Tool{} Program}
    \label{fig:syncode_error}
\end{figure}

Figure \ref{fig:syncode_error} presents an example of when the \Tool{} augmented LLM fails to generate a complete program within the maximum token limit for a problem from the HumanEval dataset. 
While the code is a syntactically correct partial program, it is not a syntactically correct complete program.  
Recall, that \Tool{} guarantees completeness for syntactically correct partial programs but does not guarantee termination with a syntactically correct complete program.

\newpage
\subsection{Prompts Used in the Evaluation}
\label{sec:prompts}
\input{prompts/json_prompts_original.tex}
\lstdefinestyle{myGrammarStyle}{
    basicstyle=\scriptsize\ttfamily, % Reduce font size
    commentstyle=\color{green},
    keywordstyle=\color{blue},
    stringstyle=\color{orange},
    numbers=left, % Line numbers on left
    numberstyle=\tiny\color{gray}, % Line numbers styling
    breaklines=true, % Wrap long lines
    frame=single, % Frame around the code
    framesep=3pt, % Adjust frame separation
    xleftmargin=5pt, % Adjust left margin
    xrightmargin=5pt, % Adjust right margin
    backgroundcolor=\color{yellow!0}, % Background color
    tabsize=2, % Tab size
    captionpos=b, % Caption position bottom
    aboveskip=5pt, % Reduce space above the listing
    belowskip=5pt, % Reduce space below the listing
    linewidth=0.9\linewidth, % Set custom line width to less than text width
    escapeinside={(*@}{@*)}, % for escaping to LaTeX
    morekeywords={Output only JSON}, % Add specific words to be highlighted
    keywordstyle=\color{blue}\bfseries, % Style for additional keywords
}

\begin{lstlisting}[style=myGrammarStyle, caption= Example JSON prompt from the JSON-Mode-Eval dataset ~\cite{jsoneval} after augmentation with an explicit request to only output JSON. ]
<s>[INST] <<SYS>>
You are a helpful assistant that answers in JSON. Here's the json schema you must adhere to:
<schema>
{'title': 'Person', 'type': 'object', 'properties': {'firstName': {'type': 'string', 'description': "The person's first name."}, 'lastName': {'type': 'string', 'description': "The person's last name."}, 'age': {'description': 'Age in years which must be equal to or greater than zero.', 'type': 'integer', 'minimum': 0}}, 'required': ['firstName', 'lastName', 'age']}
</schema>

<</SYS>>

Please generate a JSON output for a person's profile that includes their first name, last name, and age. The first name should be 'Alice', the last name 'Johnson', and the age 35. Output only JSON. [/INST]
\end{lstlisting}
\label{prompt:json_prompt_explicit}

% (*@\hl{Output only JSON.}@*)
\lstdefinestyle{myGrammarStyle}{
    basicstyle=\scriptsize\ttfamily, % Reduce font size
    commentstyle=\color{green},
    keywordstyle=\color{blue},
    stringstyle=\color{orange},
    numbers=left, % Line numbers on left
    numberstyle=\tiny\color{gray}, % Line numbers styling
    breaklines=true, % Wrap long lines
    frame=single, % Frame around the code
    framesep=3pt, % Adjust frame separation
    xleftmargin=5pt, % Adjust left margin
    xrightmargin=5pt, % Adjust right margin
    backgroundcolor=\color{yellow!0}, % Background color
    tabsize=2, % Tab size
    captionpos=b, % Caption position bottom
    aboveskip=5pt, % Reduce space above the listing
    belowskip=5pt, % Reduce space below the listing
    linewidth=0.9\linewidth, % Set custom line width to less than text width
    escapeinside={(*@}{@*)}, % for escaping to LaTeX
}

\begin{lstlisting}[style=myGrammarStyle, caption= text-2-SQL prompt.]

db_id: concert_singer
db_info: # stadium ( stadium_id , location , name , capacity , highest , lowest , average )
# singer ( singer_id , name , country , song_name , song_release_year , age , is_male )
# concert ( concert_id , concert_name , theme , stadium_id , year )
# singer_in_concert ( concert_id , singer_id )
# concert.stadium_id = stadium.stadium_id
# singer_in_concert.singer_id = singer.singer_id
# singer_in_concert.concert_id = concert.concert_id

question: How many singers do we have? Only output the SQL query. 
SQL:

\end{lstlisting}
\label{prompt:sql_prompt}
\lstdefinestyle{myGrammarStyle}{
    basicstyle=\scriptsize\ttfamily, % Reduce font size
    commentstyle=\color{green},
    keywordstyle=\color{blue},
    stringstyle=\color{orange},
    numbers=left, % Line numbers on left
    numberstyle=\tiny\color{gray}, % Line numbers styling
    breaklines=true, % Wrap long lines
    frame=single, % Frame around the code
    framesep=3pt, % Adjust frame separation
    xleftmargin=5pt, % Adjust left margin
    xrightmargin=5pt, % Adjust right margin
    backgroundcolor=\color{yellow!0}, % Background color
    tabsize=2, % Tab size
    captionpos=b, % Caption position bottom
    aboveskip=5pt, % Reduce space above the listing
    belowskip=5pt, % Reduce space below the listing
    linewidth=0.9\linewidth, % Set custom line width to less than text width
    escapeinside={(*@}{@*)}, % for escaping to LaTeX
    % morekeywords={Output, JSON}, % Add specific words to be highlighted
    % keywordstyle=\color{red}\bfseries, % Style for additional keywords
}

\begin{lstlisting}[style=myGrammarStyle, caption= Example Python prompt from the HumanEval dataset ~\cite{athiwaratkun2023multilingual}]
def has_close_elements(numbers: List[float], threshold: float) -> bool:
        """ Check if in given list of numbers, are any two numbers closer to each other than
        given threshold.
        >>> has_close_elements([1.0, 2.0, 3.0], 0.5)
        False
        >>> has_close_elements([1.0, 2.8, 3.0, 4.0, 5.0, 2.0], 0.3)
        True
        """

\end{lstlisting}
\label{prompt:python_prompt}
\lstdefinestyle{myGrammarStyle}{
    basicstyle=\scriptsize\ttfamily, % Reduce font size
    commentstyle=\color{green},
    keywordstyle=\color{blue},
    stringstyle=\color{orange},
    numbers=left, % Line numbers on left
    numberstyle=\tiny\color{gray}, % Line numbers styling
    breaklines=true, % Wrap long lines
    frame=single, % Frame around the code
    framesep=3pt, % Adjust frame separation
    xleftmargin=5pt, % Adjust left margin
    xrightmargin=5pt, % Adjust right margin
    backgroundcolor=\color{yellow!0}, % Background color
    tabsize=2, % Tab size
    captionpos=b, % Caption position bottom
    aboveskip=5pt, % Reduce space above the listing
    belowskip=5pt, % Reduce space below the listing
    linewidth=0.9\linewidth, % Set custom line width to less than text width
    escapeinside={(*@}{@*)}, % for escaping to LaTeX
}

\begin{lstlisting}[style=myGrammarStyle, caption= Example Go prompt from the HumanEval dataset ~\cite{athiwaratkun2023multilingual}]
package main

import (
	"encoding/json"
	"reflect"
)
// You're an expert Golang programmer
// Check if in given list of numbers, are any two numbers closer to each other than
// given threshold.
// >>> has_close_elements([1.0, 2.0, 3.0], 0.5)
// False
// >>> has_close_elements([1.0, 2.8, 3.0, 4.0, 5.0, 2.0], 0.3)
// True
// 
func has_close_elements (numbers []float64, threshold float64) bool {

\end{lstlisting}
\label{prompt:go_prompt}


\subsection{Grammars Used in the Evaluation}
\label{sec:grammar}

\subsubsection{JSON Grammar}
\  

\lstdefinestyle{myGrammarStyle}{
    basicstyle=\scriptsize\ttfamily, % Reduce font size
    commentstyle=\color{gray},
    keywordstyle=\color{blue},
    stringstyle=\color{orange},
    numbers=left, % Line numbers on left
    numberstyle=\tiny\color{gray}, % Line numbers styling
    breaklines=true, % Wrap long lines
    frame=single, % Frame around the code
    framesep=3pt, % Adjust frame separation
    xleftmargin=5pt, % Adjust left margin
    xrightmargin=5pt, % Adjust right margin
    backgroundcolor=\color{yellow!4}, % Background color
    tabsize=2, % Tab size
    captionpos=b, % Caption position bottom
    aboveskip=5pt, % Reduce space above the listing
    belowskip=5pt, % Reduce space below the listing
    linewidth=0.9\linewidth, % Set custom line width to less than text width
    escapeinside={(*@}{@*)}, % for escaping to LaTeX
}

\begin{lstlisting}[style=myGrammarStyle, caption=JSON Grammar]
?start: value

?value: object
| array
| UNESCAPED_STRING
| SIGNED_NUMBER      -> number
| "true"             -> true
| "false"            -> false
| "null"             -> null

array  : "[" [value ("," value)*] "]"
object : "{" [pair ("," pair)*] "}"
pair   : UNESCAPED_STRING ":" value

UNESCAPED_STRING: /\"[^"]*\"/

DIGIT: "0".."9"
HEXDIGIT: "a".."f"|"A".."F"|DIGIT
INT: DIGIT+
SIGNED_INT: ["+"|"-"] INT
DECIMAL: INT "." INT? | "." INT


_EXP: ("e"|"E") SIGNED_INT
FLOAT: INT _EXP | DECIMAL _EXP?
NUMBER: FLOAT | INT
SIGNED_NUMBER: ["+"|"-"] NUMBER
WS: /[ \t\f\r\n]/+

%ignore WS
\end{lstlisting}
\label{gram:json_grammar}


\subsubsection{SQL Grammar}
\  

\input{grammars/sql_grammar}

\subsubsection{Python Grammar}
\  

\input{grammars/python_grammar.tex}

\newpage
\subsubsection{Go Grammar}
\  

\lstdefinestyle{myGrammarStyle}{
    basicstyle=\scriptsize\ttfamily, % Reduce font size
    commentstyle=\color{gray},
    keywordstyle=\color{blue},
    stringstyle=\color{orange},
    numbers=left, % Line numbers on left
    numberstyle=\tiny\color{gray}, % Line numbers styling
    breaklines=true, % Wrap long lines
    frame=single, % Frame around the code
    framesep=3pt, % Adjust frame separation
    xleftmargin=5pt, % Adjust left margin
    xrightmargin=5pt, % Adjust right margin
    backgroundcolor=\color{yellow!4}, % Background color
    tabsize=2, % Tab size
    captionpos=b, % Caption position bottom
    aboveskip=5pt, % Reduce space above the listing
    belowskip=5pt, % Reduce space below the listing
    linewidth=0.9\linewidth, % Set custom line width to less than text width
    escapeinside={(*@}{@*)}, % for escaping to LaTeX
}

\begin{lstlisting}[style=myGrammarStyle, caption=Go Grammar]

start: package_clause eos (import_decl eos)* ((function_decl | method_decl | declaration) eos "eoc"?)*

package_clause: "package" NAME

import_decl: "import"  (import_spec | "(" (import_spec eos)* ")")

import_spec: ("." | NAME)? import_path

import_path: string_

declaration: const_decl | type_decl | var_decl

const_decl: "const"  (const_spec | "(" (const_spec eos)* ")")

const_spec: identifier_list (type_? "=" expression_list)?

identifier_list: NAME ("," NAME)*

expression_list: expression ("," expression)*

type_decl: "type" (type_spec | "(" (type_spec eos)* ")")

type_spec: alias_decl | type_def

alias_decl : NAME "=" type_

type_def : NAME type_parameters? type_

type_parameters : "[" type_parameter_decl ("," type_parameter_decl)* "]"

type_parameter_decl : identifier_list type_element

type_element : type_term ("|" type_term)*

type_term : "~"? type_

// Function declarations

function_decl: "func" NAME type_parameters? signature ("{" statement_list? ("}" | "eof"))? 

method_decl: "func" receiver NAME signature block?

receiver: parameters

var_decl: "var" (var_spec | "(" (var_spec eos)* ")")

var_spec: identifier_list (type_ ("=" expression_list)? | "=" expression_list)

block: "{" statement_list? "}"

statement_list: ((";"? | EOS?) statement eos)+

statement: declaration | labeled_stmt | simple_stmt | go_stmt | return_stmt | break_stmt | continue_stmt | goto_stmt | fallthrough_stmt | block | if_stmt | switch_stmt | select_stmt | for_stmt | defer_stmt

simple_stmt: send_stmt | inc_dec_stmt | assignment | expression | short_var_decl

send_stmt: expression  "<-" expression

inc_dec_stmt: expression ("++" | "--")

assignment: expression assign_op expression | expression_list "=" expression_list

assign_op: "+=" | "-=" | "|=" | "^=" | "*=" | "/=" | "%=" | "<<=" | ">>=" | "&=" | "&^="

short_var_decl: expression_list ":=" expression_list

labeled_stmt: NAME ":" statement?

return_stmt: "return" expression_list?

break_stmt: "break" NAME?

continue_stmt: "continue" NAME?

goto_stmt: "goto"  NAME

fallthrough_stmt: "fallthrough" 

defer_stmt: "defer" expression

if_stmt: "if"  ( expression | eos expression | simple_stmt eos expression) block ("else" (if_stmt | block))?

switch_stmt: expr_switch_stmt | type_switch_stmt

expr_switch_stmt: "switch"  (expression? | simple_stmt? eos expression?) "{" expr_case_clause* "}"

expr_case_clause: expr_switch_case ":" statement_list?

expr_switch_case: "case" expression_list | "default"

type_switch_stmt: "switch"  ( type_switch_guard | eos type_switch_guard | simple_stmt eos type_switch_guard) "{" type_case_clause* "}"

type_switch_guard: (NAME ":=")? NAME "." "(" "type"  ")"

type_case_clause: type_switch_case ":" statement_list?

type_switch_case: "case" type_list | "default"

type_list: (type_ | "nil" ) ("," (type_ | "nil"  ))*

select_stmt: "select" "{" comm_clause* "}"

comm_clause: comm_case ":" statement_list?

comm_case: "case" (send_stmt | recv_stmt) | "default"

recv_stmt: (expression_list "=" | identifier_list ":=")? expression

for_stmt: "for" [for_clause] block

for_clause: simple_stmt (eos expression eos simple_stmt)? | range_clause

range_clause: (expression_list "=" | expression_list ":=") "range"  expression

go_stmt: "go"expression

type_: literal_type | var_or_type_name type_args? | "(" type_ ")" 

channel_type

type_args : "--"

var_or_type_name: NAME "." NAME | NAME | NAME "." "(" type_ ")"

array_type: "[" array_length "]" element_type

array_length: expression

element_type: type_

pointer_type: "*" type_

interface_type: "interface" "{" ((method_spec | type_element ) eos)* "}"

slice_type: "[" "]" element_type

// It's possible to replace `type` with more restricted type_lit list and also pay attention to nil maps
map_type: "map" "[" type_ "]" element_type

channel_type: ("'chan"  | "chan"   "<-" |  "<-" "chan" ) element_type

method_spec: NAME parameters result | NAME parameters

function_type: "func" signature

signature: parameters result?

result: parameters | type_

parameters: "(" parameter_decl ("," parameter_decl)* ","? ")" | "(" ")" 

// a comma-separated list of either (a) name, (b) type, or (c) name and type 
// parameter_decl: identifier_list? "..."? type_


// Although following is overapproximate it's an easy way to avoid reduce/reduce conflicts
parameter_decl: (type_ | "..."? type_ | NAME type_)


expression: primary_expr 
            | ("+" | "-" | "!" | "^" | "*" | "&" | "<-") expression 
            | expression ("*" | "/" | "%" | "<<" | ">>" | "&" | "&^") expression 
            | expression ("+" | "-" | "|" | "^") expression 
            | expression ("==" | "!=" | "<" | "<=" | ">" | ">=") expression 
            | expression "&&" expression 
            | expression "||" expression

primary_expr: operand | primary_expr ("." (NAME | "(" type_ ")") | index | slice_ | arguments) | type_

// Giving operand higher precedence than type_ is a hack to avoid reduce/reduce conflicts
operand.3: literal | NAME | "(" expression ")" // removed NAME type_args?

literal: basic_lit | composite_lit | function_lit

basic_lit: "nil" | integer | string_ | FLOAT_LIT | CHAR_LIT

integer: DECIMAL_LIT | BINARY_LIT | OCTAL_LIT | HEX_LIT

DECIMAL_LIT: /0|[1-9]\d*/i
HEX_LIT.2: /0x[\da-f]*/i
OCTAL_LIT.2: /0o[0-7]*/i
BINARY_LIT.2 : /0b[0-1]*/i
FLOAT_LIT.2: /((\d+\.\d*|\.\d+)(e[-+]?\d+)?|\d+(e[-+]?\d+))/i
CHAR_LIT: /'.'/i

composite_lit: literal_type literal_value

literal_type: struct_type | array_type | "[" "..." "]" element_type | slice_type | map_type  | "interface" "{" "}"

literal_value: "{" (element_list ","?)? "}"

element_list: keyed_element ("," keyed_element)*

keyed_element: (key ":")? element

key: expression | literal_value

element: expression | literal_value

struct_type: "struct" "{" (field_decl eos)* "}"

field_decl: (identifier_list type_ | embedded_field) string_?

string_: RAW_STRING_LIT | INTERPRETED_STRING_LIT

RAW_STRING_LIT: /`.*?`/
INTERPRETED_STRING_LIT: /".*?"/i

embedded_field: "*"? (NAME "." NAME | NAME)  type_args?

function_lit: "func" signature block // function

index: "[" expression "]"

slice_: "[" ( expression? ":" expression? | expression? ":" expression ":" expression) "]"

type_assertion: "." "(" type_ ")"

arguments: "(" ( expression_list? "..."? ","?)? ")"

eos: ";" | EOS // | {this.closingBracket()}?
	
NAME : /[a-zA-Z_]\w*/
EOS: _NL | ";" | "/*' .*? '*/"       

COMMENT : /\/\/[^\n]*/ 
_NL: ( /(\r?\n[\t ]*)+/ | COMMENT)+

%ignore /[\t ]/
%ignore /\\[\t \f]*\r?\n/   // LINE_CONT

\end{lstlisting}
\label{gram:go_grammar}

